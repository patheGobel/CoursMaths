\documentclass[12pt]{article}
\usepackage{stmaryrd}
\usepackage{graphicx}
\usepackage[utf8]{inputenc}

\usepackage[french]{babel}
\usepackage[T1]{fontenc}
\usepackage{hyperref}
\usepackage{verbatim}

\usepackage{color, soul}

\usepackage{pgfplots}
\pgfplotsset{compat=1.15}
\usepackage{mathrsfs}

\usepackage{amsmath}
\usepackage{amsfonts}
\usepackage{amssymb}
\usepackage{tkz-tab}
\author{\\Lycée de Dindéfelo\\Mr BA}
\title{\textbf{Variables aléatoires et Probabilités conditionnelles}}
\date{\today}
\usepackage{tikz}
\usetikzlibrary{arrows, shapes.geometric, fit}

% Commande pour la couleur d'accentuation
\newcommand{\myul}[2][black]{\setulcolor{#1}\ul{#2}\setulcolor{black}}
\newcommand\tab[1][1cm]{\hspace*{#1}}

\begin{document}
\maketitle
\newpage
\section*{\underline{\textbf{\textcolor{red}{Exercice 1}}}}
Exercice 1
Dans un jeu de 32 cartes on a quatre « couleurs » : pique, trèfle, carreau et cœur ; chaque « couleur » comprend huit cartes dont une carte as.

1) On tire simultanément 3 cartes d’un jeu de 32 cartes bien battu.Calculer la probabilité de chacun des événements suivants :

  A : « les trois cartes sont des as »
  
  B : « il y a au moins 2 couleurs parmi ces 3 cartes »
  
  C : « il n’y a pas d’as parmi les 3 cartes »
  
2) On tire successivement avec remise 3 cartes du jeu de 32 cartes. Le nombre de coeurs tiré définit une variable aléatoire X. Déterminer la loi de probabilité de  X, son espérance mathématique et son écart- type
\section*{\underline{\textbf{\textcolor{red}{Exercice 2}}}}
1. On dispose d’une urne U1 contenant trois boules rouges et sept boules noires.
On extrait simultanément deux boules de cette urne, on considère que tous les tirages sont équiprobables.

a. Quelle est la probabilité P1 que les deux boules tirées soient rouges ?

    b. Quelle est la probabilité P2 que les deux boules tirées soient noires ?
    
    c. Quelle est la probabilité P3 que les deus boules tirées soient de même couleur ?
    
    d. Quelle est la probabilité P3 que les deus boules tirées soient de couleurs différentes ?
    
2. On dispose aussi d’une deuxième urne U2 contenant quatre boules rouges et six boules noires.

On tire simultanément deux boules de l’urne U1 et une boule de l’urne U2. On suppose que tous les tirages sont équiprobables. On considère les événements suivants.

R « les trois boules tirées sont rouges »   

D « les trois boules tirées ne sont pas toute de même couleur. »

B « la boule tirée dans l’urneU2 est rouge »

      a. Calculer P(R).
      
      b. Quelle est la probabilité de tirer trois boules de même couleur ?
      
      c. Calculer la probabilité de l’événement B sachant que l’événement D est réalisé. 
      
\section*{\underline{\textbf{\textcolor{red}{Exercice 3}}}}
Pour un examen dix examinateurs ont prépare chacun deux sujets. On dispose donc de vingt sujets que l’on place dans des enveloppes identiques. Deux candidats se présentent : chacun choisit au hasard deux sujets ; de plus les sujets choisis par le premier candidats ne seront plus disponibles pour le deuxième.

On note A1 l’événement : «  les deux sujets obtenus par le premier candidat provienne du même examinateur » et A2 « les deux sujets obtenus par le deuxième candidat proviennent du même examinateur ».   

1. Montrer que la probabilité de l’événement $A_{1}$ est égale à $\frac{1}{19}$

2. a. Calculer directement la probabilité conditionnelle $p(A_{2}/A_{1})$

  b. Montrer que la probabilité que les deux candidats obtiennent chacun deux sujets provenant d’un même examinateur est $\frac{1}{323}$
  
3) a. Calculer $p(A_{2}/\overline{A_{1}})$

b. En remarquant que $A_{2}=(A_{2}\cap A_{1})\cup(A_{2}\cap \overline{A_{1}})$

Calculer $p(A_{2})$ puis en déduire que $p(A_{2}\cup A_{1})=\frac{33}{323}$

4. Soit X la variable aléatoire égale au nombre de candidats qui ont choisi chacun deux sujets provenant d’un même examinateur. La variable aléatoire X prend donc les deux valeurs 0,1 ou 2.

      a. Déterminer la loi de probabilité de X.
      
      b. Calculer l’espérance mathématique de la variable aléatoire X.
\section*{\underline{\textbf{\textcolor{red}{Exercice 4}}}}
Une urne contient six boules indiscernables au toucher : Quatre boules grises et deux boules jaunes.

    1. On tire simultanément au hasard deux boules de l’urne. On note X la variable aléatoire qui à chaque tirage de deux boules associe le nombre de boules grises tirées. 
    
      Déterminer la loi de probabilité de X et calculer son espérance.
      
    2. On tire au hasard deux fois de suite deux boules simultanément, les boules n’étant pas remises dans l’urne.  On note A, B, C, D les événements suivants :
    
     A « aucune boule grise n’est tirée au cours du premier tirage de deux boules »
     
     B « une boule grise et une boule jaune sont tirées au cours du premier tirage de deux boules »
     
     C « deux boules grises sont tirées au cours du premier tirage de deux boules »
     
     D « une boule grise et une boule jaune sont tirée au cours du deuxième tirage de deux boules »
     
      a. Calculer $P (D/A)$ ; $P (D/B)$ ; $P (D/C)$
      
      b. En déduire les probabilités des événements $D\cup A$ $D\cup B$ $D\cup C$
      
      c. Calculer la probabilité de l’événement D.
\section*{\underline{\textbf{\textcolor{red}{Exercice 5}}}}
1. Un industriel produit des balances dans deux usines A et B. Pour une période donnée, l’usine A fabrique 2400 balances dont $6\%$ présentent des défauts, l’usine B fabrique 4000 balances dont $7\%$ présentent des défauts. Ces deux productions sont stockées au centre d’expédition. On prélève au hasard l’une de ces balances.

    b. Quelle est la probabilité que la balance présente un défaut.
    
    c. Quelle est la probabilité que la balance ne soit pas défectueuse?
    
    d. Sachant que la balance est défectueuse, quelle est la probabilité qu’elle soit de l’usine A ? de l’usine B ?

3. Des tests effectués sur les balances vendues ont montré que $90\%$ de ces balances fonctionnaient encore parfaitement à la fin de la garantie d’un an.

4. Un hôtelier en achète six. Notons $X$ la variable aléatoire qui compte le nombre de balances qui fonctionnent parfaitement au bout d’un an.

    a. Expliquer pourquoi la loi de probabilité de $X$ est une loi binomiale.
    
    b. Quelle est la probabilité qu’au bout d’un an :
    
        $\bullet$ Toutes les balances fonctionnent
        
        $\bullet$ Aucune balance ne fonctionne
        
        $\bullet$ Au moins la moitié des balances fonctionnent
        
        $\bullet$ Au plus la moitié des balances fonctionnent.

\section*{\underline{\textbf{\textcolor{red}{Exercice 6}}}}
Une maladie atteint $7\%$ de la population. On dispose d’un test biologique pour la détecter. Ce test donne les résultats suivants :
\begin{itemize}
    \item Chez les bien-portants, $4\%$ des réponses sont positives et $96\%$ des réponses sont négatives.
    \item Chez les malades, $92\%$ des réponses sont positives et $8\%$ des réponses sont négatives.
\end{itemize}

On décide d’hospitaliser tous les individus positifs.

On pose :
\begin{itemize}
    \item $M$ l’événement : « être malade »
    \item $T$ l’événement : « avoir le test positif »
\end{itemize}

a. Traduire les trois probabilités ci-dessus.

b. Déterminer $P(T)$.

c. Déterminer le pourcentage de bien-portants parmi les individus hospitalisés et le pourcentage d’individus malades parmi ceux qui ne sont pas hospitalisés.

\section*{\underline{\textbf{\textcolor{red}{Exercice 7}}}}
Une urne contient 10 boules : une boule porte le chiffre 0 ; trois boules portent le chiffre 1 et six boules portent le chiffre 2. On extrait simultanément 3 boules de l’urne ; on suppose que les boules ont la même chance d’être prélevées.

1) 
a) Quelle est la probabilité d’obtenir au moins une boule portant le chiffre 2 ?
b) Quelle est la probabilité d’obtenir 3 boules portant le même chiffre ?

2) On désigne par X la somme des chiffres portés par les trois boules.
a) Déterminer la loi de probabilité de X ; son espérance mathématique.
b) Définir et représenter sa fonction de répartition F.

\end{document}