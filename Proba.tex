\documentclass[12pt]{article}
\usepackage{stmaryrd}
\usepackage{graphicx}
\usepackage[utf8]{inputenc}

\usepackage[french]{babel}
\usepackage[T1]{fontenc}
\usepackage{hyperref}
\usepackage{verbatim}

\usepackage{color, soul}

\usepackage{pgfplots}
\pgfplotsset{compat=1.15}
\usepackage{mathrsfs}

\usepackage{amsmath}
\usepackage{amsfonts}
\usepackage{amssymb}
\usepackage{tkz-tab}
\author{\\Lycée de Dindéfelo\\Mr BA}
\title{\textbf{Probabilité}}
\date{\today}
\usepackage{tikz}
\usetikzlibrary{arrows, shapes.geometric, fit}

% Commande pour la couleur d'accentuation
\newcommand{\myul}[2][black]{\setulcolor{#1}\ul{#2}\setulcolor{black}}
\newcommand\tab[1][1cm]{\hspace*{#1}}

\begin{document}
\maketitle
\newpage
\section*{\underline{\textbf{\textcolor{red}{I.Vocabulaire Et Notation}}}}
\subsection*{\underline{\textbf{\textcolor{red}{1.Expérience aléatoire}}}}
Une expérience est dite aléatoire si l'issue est impossible à prevoir.\\
\subsection*{\underline{\textbf{\textcolor{red}{Exemple}}}}
Le lancer d'une pièce de monnaie est une expérience aléatoire car on ne peut pas prédire avec certitude si elle va tomber sur pile ou face.\\
\subsection*{\underline{\textbf{\textcolor{red}{2.Univers des possibles}}}}
L'unvers des possibles noté $\Omega$ désigne l'ensemble de toutes les issues possibles d'une expérience aléatoire.\\
Cet ensemble comprend tous les résultats que l'expérince pourrait produire
\subsection*{\underline{\textbf{\textcolor{red}{Exemple}}}}
\begin{itemize}
\item Dans le lancé d'un dé à six faces, l'univers des possibles est\\ 
$\Omega=\left\lbrace 1, 2, 3, 4, 5, 6\right\rbrace$, car ce sont les six valeurs que le dé peut afficher.\\
\item Dans le lancé d'un dé à six faces, l'univers des possibles est\\ 
$\Omega=\left\lbrace pile, face\right\rbrace$
\end{itemize}
\subsection*{\underline{\textbf{\textcolor{red}{3.Evènement}}}}
Soit $\Omega=\left\lbrace pile, face\right\rbrace$ l'unvers d'une expérience aléatoire.\\
On appelle évènement une partie ou un sous ensemble de
$\Omega=\left\lbrace pile, face\right\rbrace$
\subsection*{\underline{\textbf{\textcolor{red}{Exemple}}}}
Soit un dé numéroté de 1 à 6.\\
l'univers de de cette expérience est $\Omega=\left\lbrace 1, 2, 3, 4, 5, 6\right\rbrace$\\
$A=\left\lbrace 1,5,2 \right\rbrace $ est un évènement de $\Omega=\left\lbrace 1, 2, 3, 4, 5, 6\right\rbrace$ \\
$B=\left\lbrace  6,4 \right\rbrace $ est un évènement de $\Omega=\left\lbrace 1, 2, 3, 4, 5, 6\right\rbrace$\\
\subsection*{\underline{\textbf{\textcolor{red}{4.Evènement Elémentaire}}}}
Soit $\Omega$ un univers associé à l'expérience aléatoire.\\
On appelle évènement élémentaire tout singleton de $\Omega$\\
\subsection*{\underline{\textbf{\textcolor{red}{Exemple}}}}
Un dé numéroté de 1 à 6.\\
$A=\left\lbrace 1 \right\rbrace $ et $B=\left\lbrace 3 \right\rbrace $ sont des évènements élémentaires\\
\subsection*{\underline{\textbf{\textcolor{red}{5.Evènement réalisé}}}}
Un évèment est réalisé s'il contient le résultat de l'expérience.\\
\underline{\textbf{\textcolor{red}{Exemple}}}\\
Soit un dé numéroté de 1 à 6.\\
$\Omega = \left\lbrace 1, 2, 3, 4, 5, 6 \right\rbrace $.\\
$A = \left\lbrace 1, 2, 6 \right\rbrace $ est un évèvement de $\Omega$.\\
Si au cours d'un lancé le numéro 2 apparait on dit que A est réalisé.
\subsection*{\underline{\textbf{\textcolor{red}{6. Évènement certain}}}}
Soit $\Omega$ l'univers d'une expérience aléatoire. Un évènement est dit certain s'il correspond à l'ensemble de l'univers, c'est-à-dire s'il inclut toutes les issues possibles de l'expérience.
\subsection*{\underline{\textbf{\textcolor{red}{Exemple}}}}
Supposons que $\Omega = \left\lbrace 1, 2, 3, 4, 5, 6\right\rbrace $ soit l'univers des possibles lors du lancer d'un dé à six faces.\\
L'évènement avoir A:"Avoir une face inférieur à 7" est certain.
\subsection*{\underline{\textbf{\textcolor{red}{7. Évènement impossible}}}}
Un évènement est considéré comme impossible s'il correspond à l'ensemble vide, c'est-à-dire s'il ne contient aucune issue possible de l'expérience aléatoire.
\subsection*{\underline{\textbf{\textcolor{red}{Exemple}}}}
Supposons que $\Omega = \left\lbrace 1, 2, 3, 4, 5, 6 \right\rbrace  $ soit l'univers des possibles lors du lancer d'un dé à six faces.\\\\

Un événement impossible serait un événement qui ne peut pas se produire. Par exemple, si nous définissons l'événement $A = \left\lbrace 7 \right\rbrace $, cet événement est impossible car le dé ne peut afficher que des valeurs de 1 à 6 incluses.\\\\

En notation mathématique :\\
$A=\left\lbrace 7 \right\rbrace $
\subsection*{\underline{\textbf{\textcolor{red}{8. Évènement contraire}}}}

L'évènement contraire d'un évènement $A$ dans $\Omega$, noté $\overline{A}$, est l'ensemble des issues de l'expérience aléatoire qui ne sont pas dans $A$.

On dit que $\overline{A}$ est le complémentaire de A dans $\Omega$ et on note 
$\overline{A}=C^{A}_{\Omega}$
\subsection*{\underline{\textbf{\textcolor{red}{Exemple}}}}

Supposons que lors du lancer d'un dé à six faces, $A$ représente l'évènement "obtenir un nombre pair". Alors, l'évènement contraire, $\overline{A}$, serait "obtenir un nombre impair".

En notation mathématique :
\[ A = \{2, 4, 6\} \]
\[ \overline{A} = \{1, 3, 5\} \]

Ainsi, si $A$ est l'évènement "obtenir un nombre pair", alors $\overline{A}$ est l'évènement "obtenir un nombre impair".

\subsection*{\underline{\textbf{\textcolor{red}{Remarque}}}}
Si A et $\overline{A}$ sont deux évènements contraires alors:

$A\cap\overline{A}=\emptyset$ ; $A\cup\overline{A}=\Omega$

\subsection*{\underline{\textbf{\textcolor{red}{8. Évènements incompatibles}}}}

Deux événements sont dits incompatibles s'ils ne peuvent pas se produire simultanément, c'est-à-dire s'ils n'ont aucune issue en commun.

\subsection*{\underline{\textbf{\textcolor{red}{Exemple}}}}

Supposons que dans un lancer de dé à six faces, nous définissons les événements suivants :

\[ A = \{\text{lancer un nombre pair}\} = \{2, 4, 6\} \]
\[ B = \{\text{lancer un nombre impair}\} = \{1, 3, 5\} \]

Comme aucun nombre ne peut être à la fois pair et impair, les événements \(A\) et \(B\) sont incompatibles.

\subsection*{\underline{\textbf{\textcolor{red}{Remarque}}}}

Si $A$ et $B$ sont deux évènements incompatibles alors: $A\cap B$ 

\section*{\underline{\textbf{\textcolor{red}{II. Généralités sur la probabilité}}}}

\subsection*{\underline{\textbf{\textcolor{red}{1. Probabilité d'un évènement}}}}

Considérons un dé numéroté de 1 à 6.

Soit $\Omega$ l'univers de cette expérience, donné par $\Omega=\{1, 2, 3, 4, 5, 6\}$.

Supposons l'évènement $A=\{5\}$.Il y a une chance sur 6 de réaliser cette possibilité.

La probabilité de l'évènement $A$ est donc égale à $\frac{1}{6}$, que l'on note

 $P(A)=\frac{1}{6}$.

De même, considérons l'évènement $B=\{1, 6\}$. Il y a deux chances sur 6 de réaliser l'évènement $B$.

La probabilité de l'évènement $B$ est donc égale à $\frac{2}{6}$. On note $P(B)=\frac{1}{3}$.
\subsection*{\underline{\textbf{\textcolor{red}{2. Probabilité uniforme}}}}
Lorsque les évènements \textcolor{blue}{élémentaires} ont la même probabilité, on dit que on a une 
\textcolor{red}{équiprobabilité.}

Une probabilité est dite uniforme si tous les évènements élémentaire sont \textcolor{red}{équiprobables}.\\
\subsection*{\underline{\textbf{\textcolor{red}{3.Formule de probabilité uniforme}}}}
Soit $\Omega$ un univers et $p$ une probabilité définit dans cet univers.

$\Omega=\left\lbrace w_{1}, w_{2},...,w_{n}\right\rbrace $ on suppose que on a une probabilité \textcolor{green}{uniforme.}
Donc $p\left\lbrace w_{1} \right\rbrace =p\left\lbrace w_{2} \right\rbrace =...
=p\left\lbrace w_{n} \right\rbrace$\\
Comme $\Omega$ est formé des évènements $w_{1}, w_{2},...,w_{n}$ Donc


$p\left\lbrace \Omega\right\rbrace =p\left\lbrace w_{1} \right\rbrace +p\left\lbrace w_{2} \right\rbrace +...+p\left\lbrace w_{n} \right\rbrace$\\
Puisqu'il y a équiprobabilité 
$p\left\lbrace \Omega\right\rbrace =n\times p\left\lbrace w_{1} \right\rbrace$

Or $\Omega$ est un évènement certain donc $p(\Omega)=1$

Ainsi, $p\left\lbrace w_{1} \right\rbrace=\frac{1}{n}$ avec $n=card(\Omega)$, 
$p\left\lbrace w_{1} \right\rbrace=\frac{1}{card(\Omega)}$

De façon général, \[ P(A) = \frac{\text{Nombre d'issues favorables à } A}{\text{Nombre total d'issues dans } \Omega} \]
Autrement dit, \[ P(A) = \frac{\text{card} (A)}{\text{card} (\Omega)} \]
\subsection*{\underline{\textbf{\textcolor{red}{Exerice d'application}}}}
Supposons que vous lanciez un dé équilibré à six faces. Calculez la probabilité des événements suivants :

\begin{enumerate}
    \item Événement A : Obtenir un nombre pair.
    \item Événement B : Obtenir un nombre impair.
    \item Événement C : Obtenir un nombre inférieur ou égal à 3.
\end{enumerate}

\subsection*{\underline{\textbf{\textcolor{red}{Solution}}}}
Soit $\Omega$ l'univers associé a cette expérience aléatoire on a

$\Omega=\left\lbrace 1, 2, 3, 4, 5, 6 \right\rbrace $ et $card(\Omega)=6$ 
%Pour un dé équilibré à six faces, chaque face a la même probabilité d'apparaître, ce qui signifie que la probabilité de chaque nombre de 1 à 6 est $\frac{1}{6}$.

\begin{enumerate}
    \item Pour l'événement A (obtenir un nombre pair), les issues favorables sont 2, 4 et 6.
      On a $A=\left\lbrace 2, 4, 6 \right\rbrace $ et $card(A)=3$
          
     Ainsi, la probabilité de l'événement A est :
    
    \[ P(A) =\frac{card(A)}{card(\Omega)}=\frac{3}{6} = \frac{1}{2} \]
    
    \item Pour l'événement B (obtenir un nombre impair), les issues favorables sont 1, 3 et 5.On a $B=\left\lbrace 1, 3, 5 \right\rbrace $ et $card(B)=3$
    
     Ainsi, la probabilité de l'événement B est également :
    
    \[ P(B) =\frac{card(B)}{card(\Omega)}=\frac{3}{6} = \frac{1}{2} \]
    
    \item Pour l'événement C (obtenir un nombre inférieur ou égal à 3), les issues favorables sont 1, 2 et 3.On a $C=\left\lbrace 1, 2, 3 \right\rbrace $ et $card(C)=3$.
    La probabilité de l'événement C est donc :
    
    \[ P(C) =\frac{card(C)}{card(\Omega)}=\frac{3}{6} = \frac{1}{2} \]
\end{enumerate}

Ainsi, dans tous les cas, la probabilité est $\frac{1}{2}$, ce qui est conforme à la notion de probabilité uniforme pour un dé équilibré à six faces.
\subsection*{\underline{\textbf{\textcolor{red}{4. Définition de Probabilité}}}}
Soit $\Omega$ un univers associé à une expérience aléatoires.\\
On appelle probabilité sur un univers $\Omega$ toute application 
$p:\Omega \rightarrow[0;1]$ vérifiant:
\begin{enumerate}
\item $p(\Omega)=1$
\item Si A et B deux évènements de $\Omega$ et

$A\cap B=\emptyset \Longrightarrow p(A\cup B)=p(A)+p(B)$
\end{enumerate}
\underline{\textbf{\textcolor{red}{Conséquence}}}\\
\begin{enumerate}
\item $\forall A$ la probabilité de l'évènement est compris entre $[0,1]$\\
$0 \leq p(A)\leq 1$
\item $p(\emptyset)=0$
\item$p(\overline{A})=1-p(A)$
\item $p(A \cup B)=p(A)+p(B)-p(A\cap B)$\\
\underline{\textbf{\textcolor{red}{Exercice 1}}}\\
On considère l’ensemble E des entiers de 20 à 40. On choisit l’un de ses nombres au hasard.
\end{enumerate}
\begin{itemize}
\item[•] A est l’événement : « le nombre est multiple de 3 »
\item[•] B est l’événement : « le nombre est multiple de 2 »
\item[•] C est l’événement : « le nombre est multiple de 6 ».
\end{itemize}
Calculer $p(A)$, $p(B)$, $p(C)$, $p(A\cap B)$, $p(A\cup B)$, $p(A\cap C)$ et $p(A\cup C)$.

\section*{\underline{\textbf{\textcolor{red}{III. Probabilité conditionnelle et indépendance}}}}
\subsection*{\underline{\textbf{\textcolor{red}{A. Probabilité conditionnelle}}}}
\underline{\textbf{\textcolor{red}{1.Définition}}}\\
Soit $\Omega$ un univers, associe à une expérience aléatoire et $p$ la probabilité définie dans cet univers. $A$ et $B$ deux évèvenements de $\Omega$ tel que $p(B)\neq 0$

$p_{\frac{A}{B}}=p_{B}(A)=\frac{p(A\cap B)}{p(B)}=\frac{card(A\cap B)}{card(B)}$

\underline{\textbf{\textcolor{red}{Exemple}}}\\
Soit une expérience aléatoire comportant les évènements A et B. On sait que
P(A)=0,4,  P(B)=0,7 et P($A\cap B$)=0,2. Calcule $p_{B}(A)$ et $p_{A}(B)$.

Pour calculer la $1^{re}$ probabilité conditionnelle, on utilise la formule.

$p_{B}(A)=\frac{p(A\cap B)}{p(B)}$

$p_{B}(A)=\frac{0,2}{0,7}$

$p_{B}(A)=\frac{2}{7}$

Pour calculer la $2^{e}$ probabilité conditionnelle, on utilise aussi la formule. Il est important de remarquer que l’intersection est \textbf{commutative}, donc 
$p(A\cap B)=p(B\cap A)$

$p_{A}(B)=\frac{p(A\cap B)}{p(A)}$

$p_{A}(B)=\frac{0,2}{0,4}$

$p_{A}(B)=\frac{1}{2}$

$p_{A}(B)=0,5$
\subsection*{\underline{\textbf{\textcolor{red}{B.Arbre pondéré}}}}
Un arbre pondéré permet de représenter une situation probabiliste qui comporte des probabilités conditionnelles.

\underline{\textbf{\textcolor{red}{Exemple }}}\\
Dans un lycée comportant 800 élèves, $55\%$ sont des filles. Parmi les filles, $10\%$ sont des pensionnaires. Ce pourcentage est le même chez les garçons.

On choisit un élève au hasard dans ce lycée et admet que ces choix sont équiprobables.

On note F(resp. P) l'événement "l'élève choisi est une fille (resp. une pensionnaire)".

On obtient l'arbre probabiliste suivante:

\begin{tikzpicture}[level distance=3cm,
  level 1/.style={sibling distance=5cm},%Ecarte les branches des 1eme ramifications
  level 2/.style={sibling distance=4cm},%Ecarte les branches des  2eme ramifications
  %level 3/.style={sibling distance=2cm}]%Ecarte les branches des 3eme ramifications
    every node/.style={text width=2cm, align=center}]%Permet de spécifier une largeur pour chaque nœud
  \node {}
    child {node {$\overline{F}$}
     child {node {$\overline{P}$}    
      }
      child {node {$P$}    
      }
    }% 1ere branche      
    child {node {$F$}  
         child {node {$\overline{P}$}    
      }
      child {node {$P$}    
      }  
    };
\node at (-3,-1.5) [right] {$0,45$};
\node at (0.8,-1.5) [right] {$0,55$};

\node at (-5,-4) [right] {$0,90$};
\node at (-2.5,-4) [right] {$0,10$};

\node at (-0.1,-4) [right] {$0,90$};
\node at (2.5,-4) [right] {$0,10$};

\end{tikzpicture}\\
\begin{itemize}
\item La somme des probabilités des deux premières branches (c'est-à-dire issues de la racine) est 1.
\item La somme des probabilités des probabilités des branches issues du noeud F est 1.
\item La somme des probabilité des probabilité des branches issues du noeud $\overline{F}$ est 1.
\end{itemize}
\underline{\textbf{\textcolor{red}{Propriété 1}}}\\
La somme des probabilités inscrites sur les branches issues d'un même noeud est égale à 1.

\underline{\textbf{\textcolor{red}{Propriété 2}}}\\
La probabilité de l'évènement représenté par un chemin est égale au produit des probabilités inscrites sur les branches de ce chemin.

En considérant l'exemple précédent, la probabilité de l'évèvenement $F\cap \overline{P}$ est:

$p(F\cap \overline{P})=p(F)\times p_{F}(\overline{P})$

On obtient:

$p(F\cap \overline{P})=0,55\times 0,90$

$p(F\cap \overline{P})=0,495$

\underline{\textbf{\textcolor{red}{La lecture d'un arbre pondéré}}}\\
\begin{tikzpicture}[level distance=3cm,
  level 1/.style={sibling distance=5cm},%Ecarte les branches des 1eme ramifications
  level 2/.style={sibling distance=4cm},%Ecarte les branches des  2eme ramifications
  %level 3/.style={sibling distance=2cm}]%Ecarte les branches des 3eme ramifications
    every node/.style={text width=2cm, align=center}]%Permet de spécifier une largeur pour chaque nœud
  \node {}
    child {node {$\overline{A}$}
     child {node {$\overline{B}$}    
      }
      child {node {$B$}    
      }
    }% 1ere branche      
    child {node {$A$}  
         child {node {$\overline{B}$}    
      }
      child {node {$B$}    
      }  
    };
\node at (-3,-1.5) [right] {$p(\overline{A})$};
\node at (0.8,-1.5) [right] {$p(A)$};

\node at (-5,-4) [right] {$p_{\overline{A}}(\overline{B})$};
\node at (-2.2,-4) [right] {$p_{\overline{A}}(B)$};

\node at (-0.1,-4) [right] {$p_{A}(\overline{B})$};
\node at (2.8,-4) [right] {$p_{A}(B)$};

\end{tikzpicture}

La probabilité de $\overline{A}$ et $\overline{B}$ est : 
$p(\overline{A}\cap \overline{B})=p_{\overline{A}}(\overline{B})\times p(\overline{A})$\\
La probabilité de $\overline{A}$ et $B$ est : $p(\overline{A}\cap B)=
p_{\overline{A}}(B)\times p(\overline{A}) $\\
La probabilité de $A$ et $\overline{B}$ est : $p(A\cap \overline{B})=
p_{A}(\overline{B})\times p(A)$\\
La probabilité de $A$ et $B$ est : $p(A\cap B)=p_{A}(B)\times p(A)$ \\
\underline{\textbf{\textcolor{red}{Propriété 3:Probabilités totales}}}\\
Soit l'univers $\Omega$ d'une expérience aléatoire $A_{1}, A_{2},...,A_{n}$ des évènements tels que $\left\lbrace A_{1}, A_{2},...,A_{n}\right\rbrace $ forme une parition de l'univers $\Omega$.\\
Alors, pour tout événement B, on a:\\
\textcolor{blue}{$p(B)=p(B\cap A_{1})+p(B\cap A_{2})+...+p(B\cap A_{n})$}.

\textbf{\textcolor{blue}{Cette formule est appelé formule des probabilité totale}}

Et, si pour tout $p(A_{i}) \neq 0$, alors:

\textcolor{blue}{$p(B)=p(A_{1})\times p_{A_{1}}(B)+p(A_{2})\times p_{A_{2}}(B)+...+p(A_{n})\times p_{A_{n}}(B)$}
\subsection*{\underline{\textbf{\textcolor{red}{C.L'indépendance de deux évènements}}}}
\underline{\textbf{\textcolor{red}{1.Définition}}}\\
On dit que deux événement $A$ et $B$ de probabilités non nulles sont indépendants lorsque:\\
$p(A\cap B)=p(A)\times p(B)$

\underline{\textbf{\textcolor{red}{Exemple}}}\\
En considérant l'exemple précédent, montre que F et P sont deux événements indépendants.\\
\underline{\textbf{\textcolor{red}{Solution}}}\\
\textbf{Already solve but think first}
%On a:
%\begin{itemize}
%\item[•]$p(F)=0,55$
%\item[•] $p(F\cap P)=p(F)\times p_{F}(p)=0,55\times 0,10 $
%\item[•] $p(\overline{F}\cap P)=p(\overline{F})\times p_{\overline{F}}(p)=0,45\times 0,10$
%\end{itemize}
%L'ensemble $\left\lbrace F, \overline{F} \right\rbrace $ formant une partition de l'univers, on a, d'après la formule des probabilité totales :\\
%$p(P)=p(F\cap P)+p(\overline{F}\cap P)$

%Alors:
%\begin{itemize}
%\item[•] $p(F\cap P)=0,055$
%\item[•] $p(F)\times p(P)=0,55\times 0,10$
%\end{itemize}
%Les événement $F$ et $P$ sont donc indépendants
\underline{\textbf{\textcolor{red}{Exercice}}}\\
Soit $A$ et $B$ deux événements indépendantes de probabilité non nulle.

Montrer que $\overline{A}$ et $\overline{B}$ sont aussi indépendantes.
\section*{\underline{\textbf{\textcolor{red}{IV. LOIS DE PROBABILITE}}}}
\subsection*{\underline{\textbf{\textcolor{red}{A. Epreuve de Bernoulli}}}}
\underline{\textbf{\textcolor{red}{1.Définition}}}\\
On appelle une epreuve de bernoulli toute expérience aléatoire qui n'a que deux issue possible. L'un des issue est appelé succé, l'autre est appelé echec.
\underline{\textbf{\textcolor{orange}{2.Probabilté sur une epreuve de Bernoulli}}}\\
Avec une epreuve de Bernoulli, la probabilité du sucé est noté $p$ et celle de l'echec est notée $1-p$.

\underline{\textbf{\textcolor{orange}{3.Schéma de Bernoulli}}}\\
On appelle en schéma de bernoulli toute épreuve où on a n répétition qui sont identiques et indépendantes.

\underline{\textbf{\textcolor{orange}{Propriété}}}\\
Soit un schéma de Bernoulli à n épreuves où pour chaque épreuve la probabilité du succé est notée $p$ et celle de l'echec est noté 1-p. 

La probabilité d'avoir exactement k succé avec $0\leq k\leq n$.

$p_{k}=C^{k}_{n}p^{k}(1-p)^{n-k}$

k=nombre de succés

n=le nombre de répétition

\underline{\textbf{\textcolor{red}{Exercice d'application}}}

On soumet à un candidat 10 questions.

Chaque question est assortie de 3 réponses parmi lesquelles une seule est exacte.

Le candidat répond au hasard aux questions.

\begin{enumerate}
    \item Déterminez la probabilité pour que le candidat trouve 3 bonnes réponses.
    
    \item Déterminez la probabilité pour que le candidat trouve au moins une bonne réponse.
\end{enumerate}

\underline{\textbf{\textcolor{red}{Correction :}}}

\textbf{Already solve but think first}

%Pour chaque question, la probabilité que le candidat trouve la bonne réponse est $p=\frac{1}{3}$, car il choisit au hasard parmi 3 réponses dont une seule est correcte.

%Donc $1-p=\frac{2}{3}$
%\begin{enumerate}
%    \item Déterminons la probabilité pour que le candidat trouve 3 réponses.
%	$p_{k}=C_{n}^{k}p^{k}(1-p)^{n-k}$ : $k=3$, $n=10$, $p=\frac{1}{3}$ et $1-p=\frac{2}{3}$
	
%	$p_{3}=C_{10}^{3}p^{3}(1-p)^{10-3}$
%    \item Déterminons la probabilité pour que le candidat trouve au moins une bonne réponse.
    
%    Soit A:"trouver une bonne réponse" donc $\overline{A}$:"Pas de bonne réponse"
    
%    p($A$)=1-p($\overline{A}$) donc p($\overline{A}$)=1-p($A$)
    
%    $p(\overline{A})=p_{k}(k>0)=C_{n}^{k}p^{k}(1-p)^{n-k}$
    
%    $p(\overline{A})=p_{k}(k=0)=C_{10}^{0}p^{0}(1-p)^{10}$
    
%    $p(\overline{A})=p_{k}(k=0)=(\frac{2}{3})^{10}$
    
%    Or $p(A)=1-p(\overline{A})$
    
%    donc $p(A)=1-(\frac{2}{3})^{10}$
%\end{enumerate}
\subsection*{\underline{\textbf{\textcolor{red}{B. Loi Binomiale}}}}
\underline{\textbf{\textcolor{red}{1.Variables aléatoires}}}

\underline{\textbf{\textcolor{red}{Exemple}}}

On lance trois fois de suite une pièce de monnaie équilibrée.\\ On gagne 2 $\$ $ pour chaque résultat

« pile » et on perd 1 $\$ $ pour chaque résultat « face ».

1°) Quel est l’ensemble E des issues possibles ?

2°) Soit X l’application de E dans $\mathbb{R}$ qui, à chaque issue, associe le gain correspondant.

	a) Quelles sont les valeurs prises par X ?

	b) Quelle est la probabilité de l’événement « obtenir un gain de 3 $\$ $ » ? On note cette probabilité $p(X = 3)$.
	
\underline{\textbf{\textcolor{red}{Solution}}}

On obtient une nouvelle loi de probabilité sur l’ensemble des gains\\
$E’ = X(E) = \left\lbrace -3 ;0 ;3 ;6 \right\rbrace $ ; nous la nommons loi de probabilité de X :
\\
\begin{tabular}{|c|c|c|c|c|}
\hline
Gain $x_{i}$ & $X_{1}=-3$ & $X_{2}=0$ & $X_{3}=3$&$X_{4}=6$\\
\hline
Probabilité $p_{i} = p(X = x_{i}) $ &$\frac{1}{8}$ &$\frac{3}{8}$&$\frac{3}{8}$&$\frac{1}{8}$ \\
\hline
\end{tabular}
\\
\underline{\textbf{\textcolor{red}{2.Définition}}}
\begin{itemize}
\item[•] Une variable aléatoire X est une application définie sur un ensemble E muni d’une
probabilité P, à valeurs dans $\mathbb{R}$.
\item[•] X prend les valeurs $x_{1}, x_{2}, …, x_{n}$ avec les probabilités $p_{1}, p_{2}, …, p_{n}$ définies par : $p_{i} = p(X = x_{i})$.
\item[•] L’affectation des $p_{i}$ aux $x_{i}$ permet de définir une nouvelle loi de probabilité. Cette loi notée $P_{X}$, est appelée
loi de probabilité de X
\end{itemize} 
\underline{\textbf{\textcolor{red}{Remarque}}}

Soit X une variable aléatoire prenant les valeurs $x_{1}, x_{2}, …, x_{n}$ avec les probabilités $p_{1}, p_{2}, …, p_{n}$. On appelle respectivement espérance mathématique de X,
variance de X et écart-type de X , les nombres suivants :
\begin{itemize}
\item[•] L’espérance mathématique est le nombre E(X) défini par : \[E(X)=\sum_{i=1}^{n}(p_{i}x_{i})\]
\item[•] La variance est le nombre V défini par : V(X)=\[\sum_{i=1}^{n}p_{i}(x_{i}-E(X))^{2}=\sum_{i=1}^{n}p_{i}x_{i}^{2}-E(X)^{2}\]
\item[•]  L’écart-type est le nombre $\sigma$ défini par : $\sigma = \sqrt{V}$
\end{itemize}
\underline{\textbf{\textcolor{red}{Exercice:}}}

Un joueur lance un dé : si le numéro est un nombre premier, le joueur gagne une somme égale au nombre considéré (en euros) ; sinon il perd ce même nombre d’euros.

1°) Si X est le gain algébrique réalisé, donner la loi de probabilité de X et calculer son espérance mathématique et son écart-type.

2°) Le jeu est-il favorable au joueur ?

\underline{\textbf{\textcolor{red}{3. Loi Binomiale}}}

Soit un schéma de Bernoulli constitué d’une suite de n épreuves.
Soit X la variable aléatoire égale au nombre de succès obtenus, alors :\\
$P(X=k)=C_{n}^{k}\times p^{k}\times (1-p)^{n-k}$ avec $(0\leq k\leq n)$

\underline{\textbf{\textcolor{red}{Exemple}}}

Un dé cubique est mal équilibré : la probabilité d’obtenir 6 est de 1/7.

On appelle succès l’événement « obtenir 6 » et échec « obtenir un numéro différent de 6 ».

on appelle X la variable aléatoire comptant le nombre de succès à l’issue des 5 lancés. On obtient les probabilités suivantes :

$P_{0}=P(X=0)=C_{5}^{0}(\frac{1}{7})^{0}(\frac{6}{7})^{5}$

$P_{1} =0,3856$ ; $P_{2} = 0,1285$ ; $P_{3} = 0,0214$ ; $P_{4} = 0,0018$ ; $P_{5} = 0,0001$.

\underline{\textbf{\textcolor{red}{4.Espérence et Ecart-type :}}}

\begin{itemize}
\item[•] Pour une loi de Bernoulli de paramètre p, l’espérance est p et l’écart type est $\sqrt{pq}$
\item[•] Pour une loi Binomiale de paramètres n et p, l’espérance est np et l’écart type est 
$\sqrt{npq} $
\end{itemize}
\subsection*{\underline{\textbf{\textcolor{red}{5.Fonction de répartition}}}}
La fonction de répartition d'une variable aléatoire est la fonction définie sur $\mathbb{R}$ par :\\
$p(X=x_{i})=p_{i}$

\begin{equation*}
F(x)=\begin{cases}
0 & \text{si } x < x_{1}\\
p_{1} & \text{si } x_{1} \leq x < x_{2}\\
p_{1} + p_{2} & \text{si } x_{2} \leq x < x_{3}\\
\vdots \\
p_{1} + p_{2} + p_{3} + \ldots + p_{k} & \text{si } x_{k} \leq x < x_{k+1}\\
\vdots \\
p_{1} + p_{2} + p_{3} + \ldots + p_{n-1} & \text{si } x_{n-1} \leq x < x_{n}\\
1 & \text{si } x \geq x_{n}
\end{cases}
\end{equation*}
F  est une fonction en escalier.

\underline{\textbf{\textcolor{red}{Exemple}}}

En considérant le tableau suivant

\begin{tabular}{|c|c|c|c|c|}
\hline
$x_{i}$ & $0$ & $1$ & $2$ & $3$\\
\hline
$p(X = x_{i}) $ &$\frac{8}{27}$&$\frac{4}{9}$&$\frac{2}{9}$&$\frac{1}{27}$\\
\hline
\end{tabular}

Ce qui entraine

\begin{equation*}
F(x)=\begin{cases}
0 & \text{si } x < 0\\
\frac{8}{27} & \text{si } 0 \leq x < 1\\
\frac{20}{27} & \text{si } 1 \leq x < 2\\
\frac{26}{27} & \text{si } 2 \leq x < 3\\
1 & \text{si } x \geq 3
\end{cases}
\end{equation*}

\definecolor{ududff}{rgb}{0.30196078431372547,0.30196078431372547,1}
\definecolor{xdxdff}{rgb}{0.49019607843137253,0.49019607843137253,1}
\begin{tikzpicture}[line cap=round,line join=round,>=triangle 45,x=1cm,y=1cm]
\begin{axis}[
x=1cm,y=1cm,
axis lines=middle,
ymajorgrids=true,
xmajorgrids=true,
xmin=-8.580000000000005,
xmax=11.220000000000006,
ymin=-3.1299999999999963,
ymax=5.129999999999994,
xtick={-8,-7,...,11},
ytick={-3,-2,-1,0,1,2,3,4,5},]
\clip(-8.58,-3.13) rectangle (11.22,5.13);
\draw [line width=2pt] (-6,0)-- (0,0);
\draw [line width=2pt] (0,0.59)-- (1,0.57);
\draw [line width=2pt] (1,2)-- (2,2);
\draw [line width=2pt] (3,3)-- (11,3);
\draw [line width=2pt] (2,2.65)-- (2.98,2.65);
\begin{scriptsize}
\draw [fill=xdxdff] (-6,0) circle (2.5pt);
%\draw[color=xdxdff] (-5.84,0.42) node {$A$};
%\draw[color=black] (-2.92,-0.08) node {$f$};
\draw [fill=xdxdff] (0,0.59) circle (2.5pt);
%\draw[color=xdxdff] (0.16,1.02) node {$C$};
%\draw[color=black] (0.56,0.5) node {$g$};
\draw [fill=ududff] (1,2) circle (2.5pt);
%\draw[color=ududff] (1.16,2.44) node {$E$};
%\draw[color=black] (1.58,1.92) node {$h$};
\draw [fill=ududff] (3,3) circle (2.5pt);
%\draw[color=ududff] (3.16,3.44) node {$G$};
%\draw[color=black] (7.06,2.92) node {$i$};
\draw [fill=ududff] (2,2.65) circle (2.5pt);
%\draw[color=ududff] (2.16,3.08) node {$I$};
%\draw[color=black] (2.56,2.58) node {$j$};
\end{scriptsize}
\end{axis}
\end{tikzpicture}
\end{document}