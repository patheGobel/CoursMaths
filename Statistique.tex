\documentclass[12pt]{article}
\usepackage{stmaryrd}
\usepackage{graphicx}
\usepackage[utf8]{inputenc}
\usepackage[french]{babel}
\usepackage[T1]{fontenc}
\usepackage{hyperref}
\usepackage{verbatim}
\usepackage{color, soul}
\usepackage{pgfplots}
\pgfplotsset{compat=1.15}
\usepackage{mathrsfs}
\usepackage{amsmath}
\usepackage{amsfonts}
\usepackage{amssymb}
\usepackage{tkz-tab}

\author{Lycée de Dindéfelo\\Mr BA}
\title{\textbf{Statistiques}}
\date{\today}

\usepackage{tikz}
\usetikzlibrary{arrows, shapes.geometric, fit}

% Commande pour la couleur d'accentuation
\newcommand{\myul}[2][black]{\setulcolor{#1}\ul{#2}\setulcolor{black}}
\newcommand\tab[1][1cm]{\hspace*{#1}}

\begin{document}
\maketitle
\newpage
\section*{\underline{\textbf{\textcolor{red}{I Rappels : Série statistique à une variable}}}}
Soit X une série statistique quantitative
\begin{table}[h]
\centering
\begin{tabular}{|c|c|c|c|c|c|}
\hline
\textbf{Modalités} \( x_i \) & \( x_1 \) & \( x_2 \) & \( x_3 \) & \ldots & \( x_p \) \\
\hline
\textbf{Effectifs} \( n_i \) & \( n_1 \) & \( n_2 \) & \( n_3 \) & \ldots & \( n_p \) \\
\hline
\end{tabular}
\end{table}
\subsection*{\underline{\textbf{\textcolor{red}{1. Fréquence :}}}}
La fréquence de la modalité $x_{i}$ est le nombre $\frac{n_{i}}{N}$ ou $n_{i}$ est l’effectif de la modalité $x_{i}$ et $N$ l’effectif total \[N=n_{1}+n_{2}+\cdots+n_{p}=\sum_{i=1}^{p}n_{i}\] 
Remarque : \[\sum_{i=1}^{p}f_{i}=\frac{n_{1}+n_{2}+\cdots+n_{p}}{N}=1\]
\subsection*{\underline{\textbf{\textcolor{red}{2.	Moyenne :}}}}
La moyenne de cette série est le réel noté  $\overline{x}$  définie par : 

\[\overline{x}=\frac{n_{1}x_{1}+n_{2}x_{2}+\cdots+n_{p}x_{p}}{N}=\frac{\sum_{i=1}^{p}n_{i}x_{i}}{N}\]
ou \[\overline{x}=f_{1}x_{1}+f_{2}x_{2}+\cdots+f_{p}x_{p}=\sum_{i=1}^{p}f_{i}x_{i}\]
\subsection*{\underline{\textbf{\textcolor{red}{3.	Variance :}}}}
La variance de cette série est le réel positif $V(x)$ définie par :

\[V(x)=\frac{n_{1}(x_{1}-\overline{x})^{2}+n_{2}(x_{2}-\overline{x})^{2}\cdots+...+ 
n_{p}(x_{p}-\overline{x})^{2}}{N}=\frac{1}{N}\sum_{i=1}^{p}n_{i}(x_{i}-\overline{x})^{2} \]
ou
\[V(x)=f_{1}(x_{1}-\overline{x})^{2}+f_{2}(x_{2}-\overline{x})^{2}\cdots+...+ 
f_{p}(x_{p}-\overline{x})^{2}=\sum_{i=1}^{p}f_{i}(x_{i}-\overline{x})^{2}\]
ou
\[V(x)=\frac{1}{N}(n_{1}x_{1}^{2}+n_{2}x_{2}^{2}+\cdots+n_{p}x_{p}^{2})\quad\textbf{moyenne des carrés – le carré de la moyenne}\]
\subsection*{\underline{\textbf{\textcolor{red}{Remarque :}}}}
Pour effectuer un calcul de la variance, la formule (3)  qui porte le nom de formule de König est en général plus simple à utiliser. 
\subsection*{\underline{\textbf{\textcolor{red}{4.	Ecart type  }}}}
l’écart type d’une série statistique est la racine carrée de la variance on le note 

$\sigma_{X}=\sqrt{V(x)}$
\subsection*{\underline{\textbf{\textcolor{red}{5.	Exercice d’application }}}}
On donne les notes de 20 élèves à un devoir de mathématiques :
12;  9; 11; 14;  9; 8 ; 15 ; 7 ; 4 ; 18 ; 12 ; 7 ; 14 ; 12 ; 15 ; 8 ; 15 ; 11 ; 12 ; 11.

1.	Dresser le tableau des effectifs et des fréquences de cette série notée X.

2.	Calculer la moyenne, la variance et l’écart type de cette série.

3.	regrouper les notes par classe d’amplitude 4, puis calculer la moyenne, la variance et l’écart type correspondant
\subsection*{\underline{\textbf{\textcolor{red}{Solution :}}}}
1.	Tableau des effectifs et des fréquences

\begin{table}[h]
\begin{tabular}{|c|c|c|c|c|c|c|c|c|c|}
\textbf{Notes} \( x_i \) & 4 & 7 & 8 & 9 & 11 & 12 & 14 & 15 & 18 \\
\hline
\textbf{Effectifs} \( n_i \) & 1 & 2 & 2 & 2 & 3 & 4 & 2 & 3 & 1 \\
\hline
\textbf{Fréquence} \( f_i \) & 0.05 & 0.10 & 0.10 & 0.10 & 0.15 & 0.20 & 0.10 & 0.15 & 0.05 \\
\hline
\end{tabular}
\end{table}
2.	la moyenne de cette série est                                                                         

Soit $\overline{x}=\frac{224}{20}=11.2$

$\overline{x}= (4 1 +7 2 + 8 2 + 9 2 + 11 3 + 12 4 +14 2 +15 3 +18 1)$

Calcule de la variance
\[
V(x) = \left( 4^2 \times 1 + 7^2 \times 2 + 8^2 \times 2 + 9^2 \times 2 + 11^2 \times 3 + 12^2 \times 4 + 14^2 \times 2 + 15^2 \times 3 + 18^2 \times 1 \right) - 11,2^2
\]

Soit \[V(x)=\frac{2734}{20}-11,2^{2}=11,26\]
L'écart type est $\sigma_{X}=\sqrt{11,26}$ soit $\sigma_{X}=3.33$

\begin{table}[h]
\begin{tabular}{|c|c|c|c|c|}
\hline
\textbf{Classe}  & $[4 ; 8[$ & $[8 ; 12[$ & $[12 ; 16[$ & $[16 ; 20[$ \\
\hline
\textbf{Effectifs}  & $3$ & $7$  & $9$ & $1$  \\
\hline
\textbf{centre}  & $6$ & $10$ & $14$ & $18$  \\
\hline
\end{tabular}
\end{table}

\textbf{Rappel :} le centre de l’intervalle   

 $[4 ; 8[$ est le nombre $\frac{a+b}{2}$  
 
Les centres des intervalles seront considérés comme les modalités  

On trouve  $=\frac{232}{20}=11,6$, $V(X)=\frac{2896}{20}-(11,6)^{2}$ soit $V(X)=10,24$ et $\sigma_{X}=3,2$
\section*{\underline{\textbf{\textcolor{red}{II. Série statistique double}}}}
L’étude simultanée de deux caractères $X$ et $Y$ sur une même population   donne une série statistique double ${(x_{i} , y_{j}, n_{ij})}$   $1\leq i\leq  p$ et $1\leq j\leq q$

Où les $xi$ sont les modalités du caractère $X$ et les $y_{j}$ celles de $Y$ et les nij l’effectif du couple   $(x_{i} , y_{j})$.

Exemple : On relevé les notes sur $10$ de dictée $(X)$ et de calcul $(Y$) d’une classe de $CM2$ de $30$ élèves. On a obtenu les résultats suivants

\begin{table}[h]
\begin{tabular}{|c|c|c|c|c|c|c|}
\hline
$X\setminus Y$  & $2$ & $3$ & $4$ & $5$ & $7$ &  \\
\hline
$0$ & $1$ & $1$ & $2$ & $1$ & $1$ & $n_{1}=6$ \\
\hline
$1$  & $3$ & $1$ & $1$ & $1$ & $1$ & $n_{2}=7$ \\
\hline
$3$ & $0$ & $0$ & $1$ & $1$ & $1$ & $n_{3}=3$ \\
\hline
$4$  & $0$ & $1$ & $0$ & $0$ & $1$ & $n_{4}=2$ \\
\hline
$5$ & $0$ & $0$ & $1$ & $0$ & $1$ & $n_{5}=2$ \\
\hline
$6$  & $1$ & $0$ & $1$ & $1$ & $0$ & $n_{6}=3$ \\
\hline
$7$ & $1$ & $2$ & $0$ & $1$ & $0$ & $n_{7}=4$ \\
\hline
$8$  & $1$ & $0$ & $0$ & $1$ & $0$ & $n_{8}=3$ \\
\hline
  & $n_{1}=7$ & $n_{2}=5$ & $n_{3}=7$ & $n_{4}=6$ & $n_{5}=5$ & $N=30$ \\
\hline
\end{tabular}
\end{table}

Tableau de contingence, l’élément de la $i^{iéme}$ ligne et de la $j^{ième}$ colonne est l’effectif $n_{ij}$ du couple $(x_{i},y_{j})$

L’ensemble des triplets 
$\left\lbrace (x_{i},y_{i},n_{ij}) \right\rbrace^{1\leq i \leq 8}_{1\leq j \leq8} $
\subsection*{\textbf{\textcolor{red}{Première série marginale :}}}
C’est l’ensemble des couples $(x_{i},y_{j}) 1\leq i\leq p$  où les $n_{i}=\sum_{j=1}^{q}n_{ij}$ (la somme des effectifs $n_{ij}$ de la $i^{ième}$ ligne)  La première série marginale de l’exemple précédent est : 
$\left\lbrace (x_{1}, n_{1}) ; (x_{2}, n_{2}) ; …  ; (x_{8}, n_{8})\right\rbrace $   où  

$\left\lbrace (0,6) ; (1,7) ; (3,3) ; (4,2) ; (5,2) ; (6,3) ; (7,4) ; (8,3)\right\rbrace $
\section*{\underline{\textbf{\textcolor{red}{I. Définitions}}}}
\section*{\underline{\textbf{\textcolor{red}{I. Définitions}}}}
\section*{\underline{\textbf{\textcolor{red}{I. Définitions}}}}
\section*{\underline{\textbf{\textcolor{red}{I. Définitions}}}}
\section*{\underline{\textbf{\textcolor{red}{I. Définitions}}}}
\end{document}