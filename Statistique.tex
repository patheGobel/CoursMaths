\documentclass[12pt]{article}
\usepackage{stmaryrd}
\usepackage{graphicx}
\usepackage[utf8]{inputenc}
\usepackage[french]{babel}
\usepackage[T1]{fontenc}
\usepackage{hyperref}
\usepackage{verbatim}
\usepackage{color, soul}
\usepackage{pgfplots}
\pgfplotsset{compat=1.15}
\usepackage{mathrsfs}
\usepackage{amsmath}
\usepackage{amsfonts}
\usepackage{amssymb}
\usepackage{tkz-tab}

\author{Lycée de Dindéfelo\\Mr BA}
\title{\textbf{Statistiques}}
\date{\today}

\usepackage{tikz}
\usetikzlibrary{arrows, shapes.geometric, fit}

% Commande pour la couleur d'accentuation
\newcommand{\myul}[2][black]{\setulcolor{#1}\ul{#2}\setulcolor{black}}
\newcommand\tab[1][1cm]{\hspace*{#1}}

\begin{document}
\maketitle
\newpage
\section*{\underline{\textbf{\textcolor{red}{I Rappels : Série statistique à une variable}}}}
Soit X une série statistique quantitative
\begin{table}[h]
\centering
\begin{tabular}{|c|c|c|c|c|c|}
\hline
\textbf{Modalités} \( x_i \) & \( x_1 \) & \( x_2 \) & \( x_3 \) & \ldots & \( x_p \) \\
\hline
\textbf{Effectifs} \( n_i \) & \( n_1 \) & \( n_2 \) & \( n_3 \) & \ldots & \( n_p \) \\
\hline
\end{tabular}
\end{table}
\subsection*{\underline{\textbf{\textcolor{red}{1. Fréquence :}}}}
La fréquence de la modalité $x_{i}$ est le nombre $\frac{n_{i}}{N}$ ou $n_{i}$ est l’effectif de la modalité $x_{i}$ et $N$ l’effectif total \[N=n_{1}+n_{2}+\cdots+n_{p}=\sum_{i=1}^{p}n_{i}\] 
Remarque : \[\sum_{i=1}^{p}f_{i}=\frac{n_{1}+n_{2}+\cdots+n_{p}}{N}=1\]
\subsection*{\underline{\textbf{\textcolor{red}{2.	Moyenne :}}}}
La moyenne de cette série est le réel noté  $\overline{x}$  définie par : 

\[\overline{x}=\frac{n_{1}x_{1}+n_{2}x_{2}+\cdots+n_{p}x_{p}}{N}=\frac{\sum_{i=1}^{p}n_{i}x_{i}}{N}\]
ou \[\overline{x}=f_{1}x_{1}+f_{2}x_{2}+\cdots+f_{p}x_{p}=\sum_{i=1}^{p}f_{i}x_{i}\]
\subsection*{\underline{\textbf{\textcolor{red}{3.	Variance :}}}}
La variance de cette série est le réel positif $V(x)$ définie par :

\[V(x)=\frac{n_{1}(x_{1}-\overline{x})^{2}+n_{2}(x_{2}-\overline{x})^{2}\cdots+...+ 
n_{p}(x_{p}-\overline{x})^{2}}{N}=\frac{1}{N}\sum_{i=1}^{p}n_{i}(x_{i}-\overline{x})^{2} \]
ou
\[V(x)=f_{1}(x_{1}-\overline{x})^{2}+f_{2}(x_{2}-\overline{x})^{2}\cdots+...+ 
f_{p}(x_{p}-\overline{x})^{2}=\sum_{i=1}^{p}f_{i}(x_{i}-\overline{x})^{2}\]
ou
\[V(x)=\frac{1}{N}(n_{1}x_{1}^{2}+n_{2}x_{2}^{2}+\cdots+n_{p}x_{p}^{2})\quad\textbf{moyenne des carrés – le carré de la moyenne}\]
\subsection*{\underline{\textbf{\textcolor{red}{Remarque :}}}}
Pour effectuer un calcul de la variance, la formule (3)  qui porte le nom de formule de König est en général plus simple à utiliser. 
\subsection*{\underline{\textbf{\textcolor{red}{4.	Ecart type  }}}}
l’écart type d’une série statistique est la racine carrée de la variance on le note 

$\sigma_{X}=\sqrt{V(x)}$
\subsection*{\underline{\textbf{\textcolor{red}{5.	Exercice d’application }}}}
On donne les notes de 20 élèves à un devoir de mathématiques :
12;  9; 11; 14;  9; 8 ; 15 ; 7 ; 4 ; 18 ; 12 ; 7 ; 14 ; 12 ; 15 ; 8 ; 15 ; 11 ; 12 ; 11.

1.	Dresser le tableau des effectifs et des fréquences de cette série notée X.

2.	Calculer la moyenne, la variance et l’écart type de cette série.

3.	regrouper les notes par classe d’amplitude 4, puis calculer la moyenne, la variance et l’écart type correspondant
\subsection*{\underline{\textbf{\textcolor{red}{Solution :}}}}
1.	Tableau des effectifs et des fréquences

\begin{table}[h]
\begin{tabular}{|c|c|c|c|c|c|c|c|c|c|}
\textbf{Notes} \( x_i \) & 4 & 7 & 8 & 9 & 11 & 12 & 14 & 15 & 18 \\
\hline
\textbf{Effectifs} \( n_i \) & 1 & 2 & 2 & 2 & 3 & 4 & 2 & 3 & 1 \\
\hline
\textbf{Fréquence} \( f_i \) & 0.05 & 0.10 & 0.10 & 0.10 & 0.15 & 0.20 & 0.10 & 0.15 & 0.05 \\
\hline
\end{tabular}
\end{table}
2.	la moyenne de cette série est                                                                         

Soit $\overline{x}=\frac{224}{20}=11.2$

$\overline{x}= (4 1 +7 2 + 8 2 + 9 2 + 11 3 + 12 4 +14 2 +15 3 +18 1)$

Calcule de la variance
\[
V(x) = \left( 4^2 \times 1 + 7^2 \times 2 + 8^2 \times 2 + 9^2 \times 2 + 11^2 \times 3 + 12^2 \times 4 + 14^2 \times 2 + 15^2 \times 3 + 18^2 \times 1 \right) - 11,2^2
\]

Soit \[V(x)=\frac{2734}{20}-11,2^{2}=11,26\]
L'écart type est $\sigma_{X}=\sqrt{11,26}$ soit $\sigma_{X}=3.33$

\begin{table}[h]
\begin{tabular}{|c|c|c|c|c|}
\hline
\textbf{Classe}  & $[4 ; 8[$ & $[8 ; 12[$ & $[12 ; 16[$ & $[16 ; 20[$ \\
\hline
\textbf{Effectifs}  & $3$ & $7$  & $9$ & $1$  \\
\hline
\textbf{centre}  & $6$ & $10$ & $14$ & $18$  \\
\hline
\end{tabular}
\end{table}

\textbf{Rappel :} le centre de l’intervalle   

 $[4 ; 8[$ est le nombre $\frac{a+b}{2}$  
 
Les centres des intervalles seront considérés comme les modalités  

On trouve  $=\frac{232}{20}=11,6$, $V(X)=\frac{2896}{20}-(11,6)^{2}$ soit $V(X)=10,24$ et $\sigma_{X}=3,2$
\section*{\underline{\textbf{\textcolor{red}{II. Série statistique double}}}}
L’étude simultanée de deux caractères $X$ et $Y$ sur une même population   donne une série statistique double ${(x_{i} , y_{j}, n_{ij})}$   $1\leq i\leq  p$ et $1\leq j\leq q$

Où les $xi$ sont les modalités du caractère $X$ et les $y_{j}$ celles de $Y$ et les nij l’effectif du couple   $(x_{i} , y_{j})$.

Exemple : On relevé les notes sur $10$ de dictée $(X)$ et de calcul $(Y$) d’une classe de $CM2$ de $30$ élèves. On a obtenu les résultats suivants\\
\\
\begin{tabular}{|c|c|c|c|c|c|c|}
\hline
$X\setminus Y$  & $2$ & $3$ & $4$ & $5$ & $7$ &  \\
\hline
$0$ & $1$ & $1$ & $2$ & $1$ & $1$ & $n_{1}=6$ \\
\hline
$1$  & $3$ & $1$ & $1$ & $1$ & $1$ & $n_{2}=7$ \\
\hline
$3$ & $0$ & $0$ & $1$ & $1$ & $1$ & $n_{3}=3$ \\
\hline
$4$  & $0$ & $1$ & $0$ & $0$ & $1$ & $n_{4}=2$ \\
\hline
$5$ & $0$ & $0$ & $1$ & $0$ & $1$ & $n_{5}=2$ \\
\hline
$6$  & $1$ & $0$ & $1$ & $1$ & $0$ & $n_{6}=3$ \\
\hline
$7$ & $1$ & $2$ & $0$ & $1$ & $0$ & $n_{7}=4$ \\
\hline
$8$  & $1$ & $0$ & $0$ & $1$ & $0$ & $n_{8}=3$ \\
\hline
  & $n_{1}=7$ & $n_{2}=5$ & $n_{3}=7$ & $n_{4}=6$ & $n_{5}=5$ & $N=30$ \\
\hline
\end{tabular}

Tableau de contingence, l’élément de la $i^{iéme}$ ligne et de la $j^{ième}$ colonne est l’effectif $n_{ij}$ du couple $(x_{i},y_{j})$

L’ensemble des triplets 
$\left\lbrace (x_{i},y_{i},n_{ij}) \right\rbrace^{1\leq i \leq 8}_{1\leq j \leq8} $

\subsection*{\textbf{\textcolor{red}{Première série marginale :}}}
C’est l’ensemble des couples $(x_{i},y_{j}) 1\leq i\leq p$  où les $n_{i}=\sum_{j=1}^{q}n_{ij}$ (la somme des effectifs $n_{ij}$ de la $i^{ième}$ ligne)  La première série marginale de l’exemple précédent est : 
$\left\lbrace (x_{1}, n_{1}) ; (x_{2}, n_{2}) ; …  ; (x_{8}, n_{8})\right\rbrace $   où  

$\left\lbrace (0,6) ; (1,7) ; (3,3) ; (4,2) ; (5,2) ; (6,3) ; (7,4) ; (8,3)\right\rbrace $

\begin{table}[h]
\begin{tabular}{|c|c|c|c|c|c|c|c|}
\hline
$x_{i}$  & $0$ & $1$ & $3$ & $4$ & $5$ & $6$ & $7$ \\
\hline
$n_{i}$  & $6$ & $7$  & $3$ & $2$ & $2$ & $3$ & $3$  \\
\hline
\end{tabular}
\end{table}
\subsection*{\textbf{\textcolor{red}{Moyenne et variance de X :}}}
on applique les formules 
\[\overline{x}=\frac{1}{N}\sum_{i=1}^{p}n_{i}x_{i}\quad;\quad (\sigma_{X})^{2}=\frac{1}{N}\sum_{i=1}^{p}n_{i}(x_{i}-\overline{x})^{2}\]
\subsection*{\textbf{\textcolor{red}{Deuxième série marginale :}}}
c’est l’ensemble des couples $(y_{j},n_{j})1\leq j\leq q$

\[n_{j}=\sum_{i=1}^{p}n_{ij} \textbf{(la somme des effectifs $n_{ij}$ de la $j^{ième}$ colonne)} \]   
\subsection*{\textbf{\textcolor{red}{Moyenne et variance de Y:}}}
on applique les formules 
\[\overline{y}=\frac{1}{N}\sum_{j=1}^{q}n_{j}x_{j}\quad;\quad (\sigma_{y})^{2}=\frac{1}{N}\sum_{j=1}^{q}n_{j}(y_{j}-\overline{y})^{2}\]
\subsection*{\underline{\textbf{\textcolor{red}{Remarque : série statistique double injective}}}}
Une série statistique est dite injective lors que : 
\[n_{ij}=\delta_{ij}=
\left\{
\begin{array}{rcl}
0 & \textbf{si}\quad i\neq j \\
1 & \textbf{si}\quad i = j
\end{array}
\right.
\]
lors que X et Y ont le même nombre de valeur ont lui fait correspondre un tableau statistique du type 

\begin{table}[h]
\begin{tabular}{|c|c|c|c|c|}
\hline
$X$  & $X_{1}$ & $X_{2}$ & $\cdots$ & $X_{n}$ \\
\hline
$Y$  & $Y_{1}$ & $Y_{2}$ & $\cdots$ & $Y_{n}$   \\
\hline
\end{tabular}
\end{table}
Dans ce tableau chaque couple $(x_{i}, y_{i})$ a pour effectif 1

Pour une série statistique injective on a : \[\overline{x}=\frac{1}{n}\sum_{i=1}^{p}x_{i}\quad;\quad (\sigma_{X})^{2}=\frac{1}{n}\sum_{i=1}^{p}(x_{i}-\overline{x})^{2}=\frac{1}{n}\sum_{i=1}^{p}(x_{i}^{2}-\overline{x}^{2})\]

\[\overline{y}=\frac{1}{n}\sum_{i=1}^{p}y_{i}\quad;\quad (\sigma_{Y})^{2}=\frac{1}{n}\sum_{i=1}^{p}(y_{i}-\overline{y})^{2}=\frac{1}{n}\sum_{i=1}^{p}(y_{i}^{2}-\overline{y}^{2})\]
\subsection*{\underline{\textbf{\textcolor{red}{Représentation graphique d’une série double}}}}
Dans un repère orthogonal, on porte en abscisse $x_{i}$ et en ordonnées les $y_{j}$, les points 
$M_{ij}(x_{i} ; y_{j})$ auront un coefficient égal à $n_{ij}$ et on note $M_{ij}(n_{ij})$

\definecolor{ududff}{rgb}{0.30196078431372547,0.30196078431372547,1}
\definecolor{xdxdff}{rgb}{0.49019607843137253,0.49019607843137253,1}
\begin{tikzpicture}[line cap=round,line join=round,>=triangle 45,x=1cm,y=1cm]
\begin{axis}[
x=1cm,y=1cm,
axis lines=middle,
ymajorgrids=true,
xmajorgrids=true,
xmin=-1.360000000000001,
xmax=32.360000000000014,
ymin=-1,
ymax=11.96,
xtick={-3,-2,...,32},
ytick={-7,-6,...,11},]
\clip(-3.36,-7) rectangle (32.36,11.96);
\draw [line width=2pt] (1,0)-- (1,2);
\draw [line width=2pt] (0,2)-- (1,2);
\draw [line width=2pt] (0,1)-- (2,1);
\draw [line width=2pt] (2,0)-- (2,1);
\draw [line width=2pt] (3,0)-- (3,3);
\draw [line width=2pt] (0,3)-- (3,3);
\draw [line width=2pt] (4,0)-- (4,4);
\draw [line width=2pt] (0,4)-- (4,4);
\draw [line width=2pt] (4,3)-- (3,3);
\draw [line width=2pt] (3,4)-- (3,3);
\begin{scriptsize}
\draw [fill=xdxdff] (1,0) circle (2.5pt);
\draw[color=xdxdff] (0.96,-0.7) node {$x_{1}$};
\draw [fill=ududff] (1,2) circle (2.5pt);
\draw[color=ududff] (1.16,2.43) node {$M_{12}$};
\draw [fill=xdxdff] (0,2) circle (2.5pt);
\draw[color=xdxdff] (-0.52,2.33) node {$y_{2}$};
\draw [fill=xdxdff] (0,1) circle (2.5pt);
\draw[color=xdxdff] (-0.44,1.29) node {$y_{1}$};
\draw [fill=ududff] (2,1) circle (2.5pt);
\draw[color=ududff] (2.16,1.43) node {$M_{21}$};
\draw[color=ududff] (1.4,1.43) node {$M_{21}$};
\draw [fill=xdxdff] (2,0) circle (2.5pt);
\draw[color=xdxdff] (2.02,-0.7) node {$x_{2}$};
\draw [fill=xdxdff] (3,0) circle (2.5pt);
\draw[color=xdxdff] (2.86,-0.7) node {$x_{3}$};
\draw [fill=ududff] (3,3) circle (2.5pt);
\draw[color=ududff] (3.35,3.43) node {$M_{33}$};
\draw [fill=xdxdff] (0,3) circle (2.5pt);
\draw[color=xdxdff] (-0.48,3.31) node {$y_{3}$};
\draw [fill=xdxdff] (4,0) circle (2.5pt);
\draw[color=xdxdff] (4.02,-0.7) node {$x_{4}$};
\draw [fill=ududff] (4,4) circle (2.5pt);
\draw[color=ududff] (4.16,4.43) node {$M_{44}$};
\draw [fill=xdxdff] (0,4) circle (2.5pt);
\draw[color=xdxdff] (-0.4,4.39) node {$y_{4}$};
\draw [fill=xdxdff] (4,3) circle (2.5pt);
\draw[color=xdxdff] (4.4,3.43) node {$M_{43}$};
\end{scriptsize}
\end{axis}
\end{tikzpicture}

L’ensemble des points $M_{ij}(n_{ij})$ est appelé \textbf{nuage de points} de la série double   
$\left\lbrace (x_{i} , y_{j}, n_{ij}) \right\rbrace $

Le barycentre des points pondérés $( M_{ij}, n_{ij})$ est appelé \textbf{point moyen} du nuage de point $G( , )$.
Cas d’une série injective ${(x_{i} , y_{i})}$, le nuage de point est formé des points 
$M_{i} (x_{i} , y_{i})$
\subsection*{\underline{\textbf{\textcolor{red}{Ajustement linéaire}}}}
$\bullet$ \textbf{la méthode des moindres carrés} 
Lors que les points $M_{i}$ du nuage sont alignés on dit qu’il y’a une corrélation parfaite entre les grandeurs $X$ et $Y$ : ces deux grandeurs varient dans le même sens ou dans le sens contraire ; il existe alors une relation affine du type $y= ax + b$ entre les valeurs de $X$ et de $Y$ permettant de prévoir l’une des grandeurs à partir de l’autre.
Lors que les points $M_{i}$ ne sont pas alignés, on essaie de trouver une droite $(D)$ « la moins éloignée possible » des points Mi du nuage ; on dit qu’on fait un ajustement linéaire et la droite $(D)$ sera appelée droite d’ajustement.
Ceci se fait lors que les points du nuage semblent être distribués autour d’une droite.

\textbf{Droite de régression$D_{Y/X}$}

\subsection*{\textbf{\textcolor{red}{Graphe}}}

\textbf{Covariance}

La covariance d’une série double $(X,Y)$ est le réel noté $cov(X,Y)$ ou défini par $\sigma_{XY}$ 

\[\sigma_{XY}=\frac{n_{11}x_{1}y_{1}+n_{12}x_{1}y_{2}+\cdots}{N}=\frac{1}{N}\sum_{i=1}^{n}n_{ij}x_{i}y_{j}-\overline{x}\overline{y}\]
Si la série est injective on aura                                                           
\[\sigma_{XY}=\frac{x_{1}y_{1}+x_{2}y_{2}+\cdots}{N}=\frac{1}{N}\sum_{i=1}^{n}x_{i}y_{j}-\overline{x}\overline{y}=\frac{1}{N}\sum_{i=1}^{n}(x_{i}-\overline{x})(y_{i}-\overline{y})\]
\textbf{La droite $D_{Y/X}$  a pour équation $D_{Y/X}$ : $ax+( \overline{y}-a\overline{x}$ )}
\textbf{Soit $D_{Y/X} : y -\overline{y}= a(x -\overline{x})$ avec  $a = \frac{\sigma_{XY}}{\sigma_{X}^{2}}$ ou $y = ax + b$   avec $b = y-a\overline{x}$}

\textbf{Droite de régression de  $X$ en $Y$ : $D_{X/Y}$}

La droite  $D_{X/Y} :  x-\overline{x} =\alpha(y-\overline{y})$  ou $x = \alpha y +\beta $        
avec $\alpha = \frac{\sigma_{XY}}{\sigma_{Y}^{2}}$ et $\beta = \overline{x}-\beta \overline{y}$

$\bullet$ \textbf{Méthode de MAYER} 

Elle consiste à partager la série statistique double en deux séries doubles et de déterminer les points moyens $G1$ et $G2$ de chaque série.

La droite $(G1G2)$ est appelée droite d’ajustement de MAYER

\subsection*{\underline{\textbf{\textcolor{red}{Exemple}}}}
On  a fait uine enquête sur cent familles portant sur le nombre d’enfants par famille $(X)$ et le nombre de piéces d’habitation par famille $(Y)$.
Les résultats sont donnés dans le tableau ci-dessous :\\

\begin{tabular}{|c|c|c|c|c|c|c|c|}
\hline
$Y\setminus X$  & $0$ & $1$ & $2$ & $3$ & $4$ &$5$ & $Y$\\
\hline
$1$ & $8$ & $2$ & $1$ & $0$ & $0$ & $0$ &\\
\hline
$2	$  & $3$ & $11$ & $10$ & $5$ & $1$ & $0$ &\\
\hline
$3$ & $1$ & $3$ & $16$ & $13$ & $4$ & $1$ &\\
\hline
$4$  & $0$ & $1$ & $3$ & $5$ & $8$ & $4$ &\\
\hline
$X$ &  &  &  & & & &\\
\hline
\end{tabular}\\
\\
\begin{itemize}
\item[1].A l’aide de ce tableau reconstituer les séries statistiques marginales.

\item[2].On partage la série statistique en deux séries statistiques doubles $S1$ et $S2$\\
\end{itemize}
\textbf{S1}
\begin{tabular}{|c|c|c|c|c|c|}
\hline
$Y\setminus X$  & $0$ & $1$ & $2$ & $Y$ \\
\hline
$1$ & $8$ & $2$ & $1$ &   \\
\hline
$2	$  & $3$ & $11$ & $10$ &   \\
\hline
$3$ & $1$ & $3$ & $16$ & \\
\hline
$4$  & $0$ & $1$ & $3$ &   \\
\hline
$X$ &  &  &  &   \\
\hline
\end{tabular}
\textbf{S2}
\begin{tabular}{|c|c|c|c|c|c|}
\hline
$Y\setminus X$  & $3$ & $4$ & $5$ & $Y$ \\
\hline
$1$ & $0$ & $0$ & $0$ &   \\
\hline
$2	$  & $5$ & $1$ & $0$ &   \\
\hline
$3$ & $13$ & $4$ & $1$ & \\
\hline
$4$  & $5$ & $8$ & $4$ &   \\
\hline
$X$ &  &  &  &   \\
\hline
\end{tabular}\\
\\
\begin{itemize}
\item[3]. déterminer leur point moyen respectif $G1$ et $G2$ 

\item[4]. déterminer une équation de la droite $(G1G2)$

\item[5]. estimer le nombre de pièces d’habitation d’une famille ayant $6$ enfants
\end{itemize}
\subsection*{\underline{\textbf{\textcolor{red}{Coefficient de corrélation linéaire :}}}}
Le réel $r =\frac{\sigma_{XY}}{\sigma_{X}\sigma_{Y}}$ avec $\sigma_{X}$ et $\sigma_{X}$ non nuls est appelé coefficient de corrélation linéaire de X et de Y.

On a  $a=\frac{r\sigma_{Y}}{\sigma_{Y}}$ et $\alpha=\frac{r\sigma_{Y}}{\sigma_{Y}}$
\subsection*{\underline{\textbf{\textcolor{red}{Remarque :}}}}
•	Lors que r<0, X et Y varient en sens inverse, on dit qu’il y’a une corrélation négative entre X et Y.

•	Lors que r<0 X et Y varient dans le même sens, on dit qu’il y’a une corrélation positive entre X et Y.

•	Si r = 0, on ne peut trouver aucune relation affine entre X et Y ; on dit que les caractères X et Y sont indépendants.

•	Si r est proche de 1 en valeur absolue on dit qu’il y’a une bonne corrélation

•	Si r est proche de 0, on dit que la corrélation est faible.

\textbf{Exercice 1}

Dans une classe de première S de 20 élèves, on a relevé les poids (X) en kg et tailles (Y) en cm des élèves.

\begin{table}[h]
%\centering
%\caption{Poids et tailles des élèves}
\begin{tabular}{|c|c|c|c|c|c|c|c|c|c|c|}
\hline
\textbf{X (kg)} & 54 & 63 & 60 & 58 & 63 & 83 & 83 & 83 & 72 & 68 \\ 
\hline
\textbf{Y (cm)} & 165 & 183 & 178 & 168 & 175 & 180 & 173 & 180 & 173 & 179 \\ 
\hline
\end{tabular}
\end{table}

\begin{table}[h]
\centering
\caption{Poids et tailles des élèves (suite)}
\begin{tabular}{|c|c|c|c|c|c|c|c|c|c|c|}
\hline
\textbf{X (kg)} & 66 & 72 & 66 & 67 & 72 & 68 & 83 & 76 & 76 & 68 \\ 
\hline
\textbf{Y (cm)} & 175 & 180 & 168 & 171 & 173 & 178 & 183 & 168 & 173 & 178 \\ 
\hline
\end{tabular}
\end{table}

\begin{enumerate}
\item Dresser le tableau de contingence.
\item Calculer $\sigma_x$, $\sigma_y$ et $\sigma_{xy}$.
\item En déduire le coefficient de corrélation.
\item 
    \begin{enumerate}
    \item On s'intéresse maintenant à tous les élèves dont le poids est 83 kg. Combien y en a-t-il ?
    \item Calculer les fréquences: $f(173/83)$ et $f(165/83)$.
    \end{enumerate}
\item 
    \begin{enumerate}
    \item On s'intéresse maintenant à tous les élèves dont la taille est 180 cm. Combien y en a-t-il ?
    \item Calculer les fréquences $f(54/180)$ et $f(58/180)$.
    \end{enumerate}
\end{enumerate}

\section*{Exercice 2}

Une entreprise a mis au point un nouveau produit et cherche à en fixer le prix de vente. Une enquête est réalisée auprès des clients potentiels ; les résultats sont donnés dans le tableau suivant où les $y_i$ représentent le nombre d'exemplaires du produit que les clients sont disposés à acheter si le prix de vente, exprimé en milliers de francs, est $x_i$.

\[
\begin{array}{|c|c|c|c|c|c|c|c|c|}
\hline
x_i & 60 & 80 & 100 & 120 & 140 & 160 & 180 & 200 \\
\hline
y_i & 952 & 805 & 630 & 522 & 510 & 324 & 205 & 84 \\
\hline
\end{array}
\]

\subsection*{Questions}
\begin{enumerate}
    \item Calculer le coefficient de corrélation linéaire de $x$ et $y$. La valeur trouvée justifie-t-elle la recherche d’un ajustement linéaire ?
    \item Déterminer l’équation de la droite de régression de $y$ en $x$.
\end{enumerate}
\end{document}