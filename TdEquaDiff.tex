\documentclass[12pt]{article}
\usepackage{stmaryrd}
\usepackage{graphicx}
\usepackage[utf8]{inputenc}

\usepackage[french]{babel}
\usepackage[T1]{fontenc}
\usepackage{hyperref}
\usepackage{verbatim}

\usepackage{color, soul}

\usepackage{pgfplots}
\pgfplotsset{compat=1.15}
\usepackage{mathrsfs}

\usepackage{amsmath}
\usepackage{amsfonts}
\usepackage{amssymb}
\usepackage{tkz-tab}
\author{Destiné aux élèves de Terminale S\\Lycée de Dindéfelo\\Présenté par M. BA}
\title{\textbf{TD Sur Les Equations Différentielles}}
\date{\today}
\usepackage{tikz}
\usetikzlibrary{arrows, shapes.geometric, fit}

% Commande pour la couleur d'accentuation
\newcommand{\myul}[2][black]{\setulcolor{#1}\ul{#2}\setulcolor{black}}
\newcommand\tab[1][1cm]{\hspace*{#1}}

\begin{document}
\maketitle
\newpage
\subsection*{Généralités. Équations du premier ordre}
\subsubsection*{Exercice 1}
\begin{itemize}
    \item Montrer que chacune des fonctions \(f\) est solution de l'équation différentielle \((E)\):
    \begin{itemize}
        \item \(f(x)=\cos(2x+3)\quad(E)\ :\  y''=16y\)
        \item \(f(x)=\sin(2x)\quad(E)\ :\  y''+3y=-2\sin x\cos x\)
        \item \(f(x)=x\mathrm{e}^{x}\quad(E)\ :\  y'-y=\mathrm{e}^{x}\)
        \item \(f(x)=\mathrm{e}^{x}\ln x\quad(E)\ :\  y''-2y'+y=\dfrac{-1}{x^{2}}\mathrm{e}^{x}\)
    \end{itemize}
\end{itemize}
\subsubsection*{Exercice 2}
\begin{itemize}
    \item Déterminer la solution \(f\) de chacune des équations différentielles \((E)\) suivantes vérifiant la condition \(f(x_{0})=y_{0}\):
    \begin{itemize}
        \item \((E)\ :\ -3y'+2y=0\;,\text{ avec }x_{0}=3\text{ et }y_{0}=1.\)
        \item \((E)\ :\ 3y+6y=0\;,\text{ avec }x_{0}=-4\text{ et }y_{0}=2\)
        \item \((E)\ :\ 5y'+y=0\;,\text{ avec }x_{0}=-5\text{ et }y_{0}=1.\)
        \item \((E)\ :\ 2y-5y'=0\;,\text{ avec }x_{0}=1\text{ et }y_{0}=-3\)
    \end{itemize}
\end{itemize}

\subsection*{Équations du second ordre}
\subsubsection*{Exercice 3}
\begin{itemize}
    \item Déterminer la solution \(f\) de chacune des équations différentielles \((E)\) suivantes vérifiant les conditions \(f(x_{0})=y_{0}\text{ et }f'(x_{0})=y'\):
    \begin{itemize}
        \item \(2y''-3y'-2y=0\;,\ f(0)=2\text{ et }f'(0)=3\)
        \item \(y''+y'+y=0\;,\ f(0)=1\text{ et }f'(0)=-1\)
        \item \(4y''-4y'+y=0\;,\ f(0)=-3\text{ et }f'(0)=2\)
        \item \(y''-5y'+6y=0\;,\ f(0)=0\text{ et }f'(0)=6\)
        \item \(9y''+6y'+y=0\;,\ f(0)=1\text{ et }f'(0)=2\)
        \item \(y''-4y'+5y=0\;,\ f(0)=-1\text{ et }f'(0)=3\)
    \end{itemize}
\end{itemize}
\subsubsection*{Exercice 4}
\begin{itemize}
    \item Soit \((E)\) l'équation différentielle du second ordre :
    \[y''-3y'+2y=0.\]
    \begin{itemize}
        \item a) Quelles sont les solutions de \((E)\) ?
        \item b) Quelle est la solution de \((E)\) dont la courbe représentative \(\mathcal{C}\) admet au point d'abscisse \(x=0\) la même tangente que la courbe \(\mathcal{C'}\) représentative de \(y=x\) ?
        \item c) Préciser les positions relatives de \(\mathcal{C}_{\lambda}\) et \(\mathcal{C'}_{\lambda}.\)
    \end{itemize}
\end{itemize}
\subsection*{Exercice 5}
\begin{enumerate}
    \item Résoudre l'équation différentielle :
    \[y''+16y=0.\]
    \item Trouver la solution \(f\) de cette équation vérifiant :
    \[f(0)=1\text{ et }f'(0)=4.\]
    \item Trouver deux réels positifs \(\omega\) et \(\varphi\) tels que pour tout réel \(t\), \(f(t)=\sqrt{2} \cos(\omega t-\varphi)\).
    \item Calculer la valeur moyenne de \(f\) sur l'intervalle \(\left[0\;,\ \dfrac{\pi}{8}\right]\).
\end{enumerate}

\subsection*{Équations différentielles linéaires du second ordre avec un second membre non nul}
\subsubsection*{Exercice 6}
On considère l'équation \((E)\ :\ y''+m y'+p y =g(x)\) où \(m\) et \(p\) sont deux réels et \(g\) une fonction continue sur \(\mathbb{R}\).

\subsubsection*{I. Étude générale}
On suppose qu'il existe une solution \(f_{1}\) de \((E)\).
\begin{enumerate}
    \item Prouver que si \(f\) est solution de \((E)\), alors \(f-f_{1}\) est solution de l'équation différentielle :
    \[(E_{0})\ :\ y''+m y'+p y=0.\]
    \item Prouver que si \(h\) est solution de \((E_{0})\), alors \(h+f_{1}\) est solution de \((E)\).
    \item En déduire toutes les solutions de \((E)\) si on connaît une solution de \((E)\).
\end{enumerate}
\subsection*{II. Cas où \(g\) est un polynôme}
\begin{enumerate}
    \item On considère l'équation \((E)\ :\ y''-3y'+2y=x+1.\)
    \begin{enumerate}
        \item Déterminer \(a\) et \(b\) réels tels que :
        \(f_{1}\ :\ x\mapsto ax+b\) soit solution de \((E).\) 
        \item Résoudre \((E).\) 
    \end{enumerate}
    \item On considère l'équation \((E)\ :\ y''-3y'+2y=x^{2}+2x+3.\)
    \begin{enumerate}
        \item Déterminer \(a\) et \(b\) réels tels que :
        \(f_{1}\ :\ x\mapsto ax^{2}+bx+c\) soit solution de \((E).\) 
        \item Résoudre \((E).\) 
    \end{enumerate}
    Soit \(P\) une fonction polynômiale du second degré.
    \item Résoudre :
    \begin{enumerate}
        \item \(y''-8y'+17y=x^{2}-x+2.\)
        \item \(y''+4y'+4y=x^{2}+1.\)
    \end{enumerate}
\end{enumerate}

\subsection*{III. Cas où \(g(x)=\alpha\cos(\omega x)+\beta\sin(\omega x)\)}
\begin{enumerate}
    \item On considère l'équation \((E)\ :\ y''+4y'+5y=2\cos 3x-\sin 3x.\)
    \begin{enumerate}
        \item Déterminer \(a\) et \(b\) réels tels que :
        \(f_{1}\ :\ x\mapsto a\cos 3x+b\sin 3x\) soit solution de \((E).\) 
        \item Résoudre \((E).\) 
    \end{enumerate}
    \item On considère l'équation \((E)\ :\ y''+4y'+5y=\alpha\cos 3x+\beta\sin 3x.\)
    \begin{enumerate}
        \item Déterminer \(a\) et \(b\) réels tels que :
        \(f_{1}\ :\ x\mapsto a\cos 3x+b\sin 3x\) soit solution de \((E).\) 
        \item Résoudre \((E).\) 
    \end{enumerate}
    \item Résoudre :
    \begin{enumerate}
        \item \(y''-6y'+8y=\cos x+2\sin x.\)
        \item \(y''+4y'+4y=\sin 5x.\)
    \end{enumerate}
\end{enumerate}

\subsection*{IV. Cas où \(g(x)=\mathrm{e}^{a x}\)}
\begin{enumerate}
    \item On considère l'équation \((E)\ :\ y''-4y'+4y=\mathrm{e}^{-2\,x}.\)
    \begin{enumerate}
        \item Déterminer \(a\) réel tels que :
        \(f_{1}\ :\ x\mapsto a\mathrm{e}^{-2\,x}\) soit solution de \((E).\) 
        \item Résoudre \((E).\) 
    \end{enumerate}
    \item On considère l'équation \((E)\ :\ y''-5y'+6y=\mathrm{e}^{2\,x}.\)
    \begin{enumerate}
        \item Peut-on déterminer une solution particulière de \((E)\) sous la forme \(x\mapsto a\mathrm{e}^{2\,x}\) ?
        \item Déterminer \(a\) et \(b\) réels tels que :
        \(f_{1}\ :\ x\mapsto(ax+b)\mathrm{e}^{2\,x}\) soit solution de \((E).\) 
        \item Résoudre \((E).\) 
    \end{enumerate}
    \item On considère l'équation \((E)\ :\ y''-4y'+4y=\mathrm{e}^{2\,x}.\)
    \begin{enumerate}
        \item Peut-on déterminer une solution particulière de \((E)\) sous la forme \(x\mapsto a\mathrm{e}^{2\,x}\) ?
        \item Peut-on déterminer une solution particulière de \((E)\) sous la forme \(x\mapsto (ax+b)\mathrm{e}^{2\,x}\) ?
        \item Déterminer \(a\text{ et }b\) réels tels que :
        \(f_{1}\ :\ x\mapsto(a\,x^{2}+bx+c)\mathrm{e}^{2\,x}\) soit solution de \((E).\) 
        \item Résoudre \((E).\) 
    \end{enumerate}
    \item On considère l'équation \((E)\ :\ y''+m y'+p y=\mathrm{e}^{a\,x}.\)
    \begin{enumerate}
        \item Prouver que si \(a\) n'est pas solution de \(r^{2}+m r+p=0\), il existe une solution du type \(x\mapsto a\mathrm{e}^{a\,x}.\)
        \item Prouver que si \(a\) est une racine simple de \(r^{2}+m r+p=0\), il n'existe pas de solution du type \(x\mapsto a\mathrm{e}^{a\,x}\) mais une solution du type \(x\mapsto(ax+b)\mathrm{e}^{a\,x}\)
        \item Prouver que si \(a\) est une racine double de \(r^{2}+m r+p=0\), il n'existe pas de solution du type \(x\mapsto a\mathrm{e}^{a\,x}\), ni du type \(x\mapsto (ax+b)\mathrm{e}^{a\,x}\), mais une solution du type \(x\mapsto (a\,x^{2}+bx+c)\mathrm{e}^{a\,x}\)
        \item Résoudre :
        \begin{enumerate}
            \item \(y''-2y'+2y=\mathrm{e}^{3\,x}\)
            \item \(y''+2y'-3y=\mathrm{e}^{x}.\)
        \end{enumerate}
    \end{enumerate}
\end{enumerate}
\end{document}