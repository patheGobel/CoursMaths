\documentclass[12pt]{article}
\usepackage{stmaryrd}
\usepackage{graphicx}
\usepackage[utf8]{inputenc}

\usepackage[french]{babel}
\usepackage[T1]{fontenc}
\usepackage{hyperref}
\usepackage{verbatim}

\usepackage{color, soul}

\usepackage{pgfplots}
\pgfplotsset{compat=1.15}
\usepackage{mathrsfs}

\usepackage{amsmath}
\usepackage{amsfonts}
\usepackage{amssymb}
\usepackage{tkz-tab}
%\author{\\Lycée de Dindéfelo\\Mr BA}
\title{\textbf{Similitude}}
\date{\today}
\usepackage{tikz}
\usetikzlibrary{arrows, shapes.geometric, fit}

% Commande pour la couleur d'accentuation
\newcommand{\myul}[2][black]{\setulcolor{#1}\ul{#2}\setulcolor{black}}
\newcommand\tab[1][1cm]{\hspace*{#1}}

\begin{document}
\maketitle
\newpage
\section*{\underline{\textbf{\textcolor{red}{I. Définitions}}}}
On appelle similitude plane directe toute transformation du plan qui multiplie les distances par un réel $k > 0$ et qui conserve la mesure des angles orientés. $k$ est appelé le rapport de la similitude.
\begin{itemize}
    \item Les translations et les rotations sont des similitudes de rapport $1$.
    \item L'homothétie de rapport $k$ est une similitude de rapport $|k|$.
    \item Les composées d'homothétie et de rotation sont des similitudes.
    \item Les rotations, les homothéties, les composées d'homothétie et de rotation sont appelées des similitudes à centre.
    \item La réciproque d'une similitude de rapport $k$ est une similitude de rapport $k^{-1}$.
    \item La composée de deux similitudes $s$ et $s'$ de rapports respectifs $k$ et $k'$ est une similitude de rapport $kk'$.
    \item Si la similitude est la composée d'une rotation $r(\Omega,\theta)$ et d'une homothétie de centre $\Omega$ et de rapport $k > 0$, on dira que $S = h \circ r$ est une similitude de rapport $k$, d'angle $\theta$ et de centre $\Omega$ notée $s(\Omega,k,\theta)$.
\end{itemize}
\subsection*{\underline{\textbf{\textcolor{red}{II. Expression complexe}}}}
Si $s$ est une similitude définie par $z' = az + b$, alors :
\begin{itemize}
    \item Si $a = 1$, alors $s$ est une translation de vecteur $\overrightarrow{u}(b)$.
    \item Si $a \in \mathbb{R}^*_+ \backslash \{1\}$, alors $s$ est une homothétie de rapport $a$ et de centre $\Omega$ tels que $z_\Omega = az_\Omega + b$.
    \item Si $a \in \mathbb{C} \backslash \mathbb{R}$ et $|a| = 1$, alors $s$ est une rotation d'angle $\theta$ tel que $\theta = \arg(a)$ et de centre $\Omega$ tels que $z_\Omega = \frac{b}{1-a}$.
    \item Si $a \in \mathbb{C} \backslash \mathbb{R}$ et $|a| = 1$, alors $z = ke^{i\theta}z + b$ où $k = |a|$ et $\theta = \arg(a)$. On dira que $s$ est une similitude de rapport $k$, d'angle $\theta$ et de centre $\Omega$ tels que $z_\Omega = \frac{b}{1-a}$.\\
%$s = h(\Omega,|a|) \circ r_{(\Omega,\theta)}$.
    \item Soit $s = s(\Omega,k,\theta)$ définie par $z - z_\Omega = ke^{i\theta} (z - z_\Omega)$ ou $z' = ke^{i\theta} z + \alpha$.
  Sa réciproque $s^{-1}$ est donnée par $s^{-1} = s_{(\Omega, \frac{1}{k} ,- \theta)}$.\\
    \item Considérons deux similitudes $s_1$ et $s_2$ définies par :
    \begin{align*}
        s_1 &= s_{(\Omega,k_1 ,\theta_1)} ; z' = k_1 e^{i\theta_1} z + \alpha_1 \\
        s_2 &= s_{(\Omega,k_2 ,\theta_2)} ; z' = k_2 e^{i\theta_2} z + \alpha_2
    \end{align*}
\end{itemize}
\textbf{Leur composée notée $s_1 \circ s_2$ sera définie par}
\[ z' = k_1 e^{i\theta_1} (k_2 e^{i\theta_2} z + \alpha_2) + \alpha_1 = k_1 k_2 e^{i(\theta_1 + \theta_2)} + \alpha \]

\begin{itemize}
    \item Si $\theta_1 + \theta_2 = 0[2\pi]$ et $k_1 k_2 = 1$ alors, $s_1 \circ s_2 = t$
    \item Si $\theta_1 + \theta_2 = 0[2\pi]$ et $k_1 k_2 \neq 1$ alors, $s_1 \circ s_2 = h$ de rapport $k_1 k_2$
    \item Si $\theta_1 + \theta_2 \neq 0[2\pi]$ et $k_1 k_2 = 1$ alors, $s_1 \circ s_2 = r$ d’angle $\theta_1 + \theta_2$
    \item Si $\theta_1 + \theta_2 \neq 0[2\pi]$ et $k_1 k_2 \neq 1$ alors, $s_1 \circ s_2 = r$ est une similitude de rapport $k_1 + k_2$, d’angle $\theta_1 + \theta_2$ et de centre $\Omega$
\end{itemize}
\subsection*{\underline{\textbf{\textcolor{red}{Exercice d’application}}}}
Le plan est muni d’un repère $(O; \vec{I}, \vec{J})$ orthonormé. Déterminer les écritures complexes des transformations du plan suivantes :
\begin{enumerate}
    \item La translation $t$ de vecteur $\overrightarrow{u}(3, -2)$.
    \item L’homothétie $h$ de centre $A(-2, 1)$ et de rapport $\frac{3}{2}$.
    \item L’homothétie de centre $A(2, 3)$ qui transforme $B(-1, -3)$ en $C(1, 1)$.
    \item L’homothétie $h$ qui applique le point $A(1, 2)$ sur le point $B(-1, -3)$ et le point $C(-2, 3)$ sur le point $D(5, -5)$.
    \item La rotation $r$ de centre $O$ et d’angle orienté $\frac{2\pi}{3}$.
    \item La rotation $r$ de centre $A(2; -2)$ qui transforme le point $B(3; -2)$ en $C(1; -2)$.
    \item La similitude directe $s$ de centre $\Omega(1 - i)$ de rapport $2$ et d’angle orienté $-\frac{5\pi}{6}$.
    \item La similitude de centre $A(-1)$ qui transforme le point $B(-1 - i)$ en $C(-i)$.
    \item La similitude qui transforme les points $A$ et $B$ respectivement en $C$ et $D$ avec $z_A = 3 + i$, $z_B = -1 + 2i$, $z_C = 6 - 3i$ et $z_D = 3 + 2i$.
\end{enumerate}
\subsection*{\underline{\textbf{\textcolor{red}{III. Propriétés}}}}
Soient $s = s_{(\Omega,k,\theta)}$ une similitude, $M$ et $N$ deux points du plan d'images respectives, par $s$, $M'$ et $N'$, alors :
\begin{itemize}
\item[•]
\[
s(M)=M'\Leftrightarrow \begin{cases}
    \Omega M' = k\Omega M \\
    (\overrightarrow{\Omega M'},\overrightarrow{\Omega M})= \theta[2\pi]
\end{cases}
\]
\item[•]
\[
\begin{cases}
    s(M) = M' \\
    s(N) = N'
\end{cases}
\Leftrightarrow 
\begin{cases}
   M'N'=kMN \\
   (\overrightarrow{MN},\overrightarrow{M'N'})=\theta[2\pi]
\end{cases}\]
\item[•] Toute similitude de rapport k peut être décomposée en la composée
d’une homothéties de rapport k et d’un déplacement.
\item[•] Les similitudes conservent les barycentres, les angles orientés, le pa-
rallélisme et la perpendicularité.
\item[•] L’image d’une figure quelconque est une figure de même nature.
\end{itemize}
\end{document}