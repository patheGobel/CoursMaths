\documentclass[12pt]{article}
\usepackage{stmaryrd}
\usepackage{graphicx}
\usepackage[utf8]{inputenc}
\usepackage[french]{babel}
\usepackage[T1]{fontenc}
\usepackage{hyperref}
\usepackage{verbatim}
\usepackage{color,soul}
\usepackage{amsmath}
\usepackage{amsfonts}
\usepackage{amssymb}
\usepackage{systeme}
\usepackage{tkz-tab}
\author{Destiné à la TerminaleS2\\Au Lycée de Dindéferlo}
\title{\textbf{Nombres Complexes}}
\date{\today}
\usepackage{tikz}
\usetikzlibrary{arrows}
\usepackage[a4paper,left=20mm,right=20mm,top=15mm,bottom=15mm]{geometry}
\usepackage{mathtools}
\usepackage{systeme}

\DecimalMathComma

\begin{document}

\maketitle
\newpage

\section*{Exercice 1}
\subsection*{1. Mettre sous forme algébrique les nombres complexes suivants :}
a. \(z_1 = (1 - i)(5 + i)\)\quad
b. \(z_2 = (2 - 3i)^2\)\quad
c. \(z_3 = \frac{1}{3 + 2i}\)\quad
d. \(z_4 = \frac{4 - 5i}{3 + 2i}\)

\subsection*{2. Écrire en fonction de $\overline{z}$ les conjugués des nombres complexes suivants:}
a. \(z_1 = 1 + iz\)\quad
b. \(z_2 = i(z + 3)\)\quad
c. \(z_3 =  \frac{1 - z}{1 + iz}\)\quad
d. \(z_4 =\frac{ 1 + 3z}{i + 2z}\)

\subsection*{3. Déterminer un argument de z dans chacun des cas suivants:}
a. \(z = -1 + i\)\quad
b. \(z = \sqrt{6} + i\sqrt{6}\)\quad
c. \(z = \frac{1}{2} - i\frac{\sqrt{3}}{2}\) \quad
d. \(z = (2 + 2i)(1 - i)\)\quad\\
e. \(z = \frac{-1 + i\sqrt{3}}{1 + i}\)\quad
f. \(z = (-1 - i)^4\)

\section*{Exercice 2}
Le plan est muni d’une repère orthonormé direct.
\subsection*{1. Déterminer puis construire l’ensemble des points M du plan d’affixe z vérifiant :}
a. \(|z - 3| = |z + i|\)\quad
b. \(|iz + 3| = |z + 4 + i|\)\quad
c. \(|\overline{z} + \frac{1}{3}| = 3\)\quad
d. \(|z - \overline{z} + i| = 2\)\quad
e. \(|\overline{z} - 2 + i| = |z + 5 - 2i|\)\quad
f. \(|\overline{z} - 2 + i| = |z + 5 - 2i|\)\quad

\subsection*{2. Pour tout nombre complexe $ z \neq -1 + 2i $, :}
on pose \(Z =\frac{z - 2 + 4i}{z + 1 - 2i}\)\\
Déterminer l’ensemble des points M du plan tels que\\
a. \(|Z| = 1\)\quad
b. \(|Z| = 2\)\quad \\
c. \(Z\) soit un réel.\\
d. \(Z\) est un imaginaire pur.\\

\subsection*{3. Pour tout complexe z $\neq$ i, on pose \(U = \frac{z + i}{z - i}\)}
Déterminer l’ensemble des points M d'affixe z tels que :\\
a. \(U \in \mathbb{R}^{*}_{-}\)\quad
b. \(U \in \mathbb{R}^{*}_{+}\)\quad
c. \(U \in i\mathbb{R}\)\quad

\section*{Exercice 3}
Le plan est muni d'une repère orthonormé direct.\\
Soit le nombre complexe	$z=\frac{2(-1 + i\sqrt{3})}{1 + i\sqrt{3}}$\\
1. Déterminer Re(z) et Im(z).\\
2. Déterminer le module et un argument de z.\\
3. En déduire le module et un argument de :\\
$\frac{1}{z}$ ; $\frac{i}{z}$ et $\frac{1+i}{z}$

\section*{Exercice 4}
1. On pose $z_{1}=\frac{\sqrt{6}+i\sqrt{2}}{2}$ ; $z_{2}=1-i$ et $z_{3}=\frac{z_{1}}{z_{2}}$.\\
	a. Déterminer un argument de $z_{1}$ ; $z_{2}$ et $z_{3}$.\\
	b. En déduire les valeurs exactes de $\cos(\frac{5\pi}{12})$ et $\sin(\frac{5\pi}{12})$\\
2. On considère les nombres complexes : $a = 1-i$ ; $b=1-i\sqrt{3}$ ; $Z=\frac{a^{5}}{b^{4}}$.
	a. Déterminer une écriture trigonométrique de Z.\\
	b. Déterminer une écriture cartésienne de Z.\\
	En déduire les valeurs de $\cos(\frac{\pi}{12})$ et $\sin(\frac{\pi}{12})$\\
	c. Calculer $Z^{12}$ et $Z^{2024}$\\
	d. Pour quelles valeurs de l’entier naturel n:\\
	$Z^{n}$ est un réel.\\
	$Z^{n}$ est un imaginaire pur

\section*{Exercice 5}
On donne $u=\sqrt{2-\sqrt{2}}+i \sqrt{2+\sqrt{2}}$\\
1. Calculer $u^{2}$ et $u^{4}$ sous forme algébrique.\\
2. En déduire le module et un argument de u.\\
3. Soit M le point d’affixe $z \in \mathbb{C}$.
Déterminer l’ensemble des points M tels que |uz| = 8
\section*{Exercice 6}
\section*{Exercice 7}
\section*{Exercice 8}
\section*{Exercice 9}
\section*{Exercice 10}
\section*{Exercice 11}
\section*{Exercice 12}
\section*{Exercice 13}
\section*{Exercice 14}
\end{document}