\documentclass[12pt]{article}
\usepackage{stmaryrd}
\usepackage{graphicx}
\usepackage[utf8]{inputenc}

\usepackage[french]{babel}
\usepackage[T1]{fontenc}
\usepackage{hyperref}
\usepackage{verbatim}

\usepackage{color, soul}

\usepackage{pgfplots}
\pgfplotsset{compat=1.15}
\usepackage{mathrsfs}

\usepackage{amsmath}
\usepackage{amsfonts}
\usepackage{amssymb}
\usepackage{tkz-tab}
\author{\\Lycée de Dindéfelo\\Mr BA}
\title{\textbf{Td Similitude}}
\date{\today}
\usepackage{tikz}
\usetikzlibrary{arrows, shapes.geometric, fit}

% Commande pour la couleur d'accentuation
\newcommand{\myul}[2][black]{\setulcolor{#1}\ul{#2}\setulcolor{black}}
\newcommand\tab[1][1cm]{\hspace*{#1}}

\begin{document}

\begin{minipage}{0.5\textwidth}
	Ministère de l'éducation nationale  \\
	Inspection académique de Kédougou   \\
	Lycée de Dindéfelo            \\
	Cellule de mathématiques            \\
	M. BA                          \\
	Classe : TS2  \\
\end{minipage}
\begin{minipage}{0.5\textwidth}
	Année scolaire 2023-2024 \\
\end{minipage}

\begin{center}
	\textbf{{\underline{TD de Prbabilité}}}
\end{center}
\section*{EXERCICE 1}
Une urne contient 15 boules identiques numérotées de 1 à 15. On tire une boule au hasard et on note $N$ son numéro. Soient les événements :
\begin{itemize}
   \item $A$ : $\ll N$ est divisible par 2 $\gg$
   \item $B$ : $\ll N$ est divisible par 3 $\gg$
\end{itemize}
Calculer les probabilités des événements $A$, $B$, $A \cup B$, $A \cap B$, $\overline{A}$, $\overline{B}$ ,$\overline{A} \cap \overline{B}$ ,$\overline{A} \cup \overline{B}$.

\section*{EXERCICE 2 : Chaussettes}
Dans le tiroir de son armoire, Abdou possède 5 paires de chaussettes noires, 3 paires de chaussettes vertes et 2 paires de chaussettes rouge. Ces chaussettes sont mélangées dans le plus grand désordre et indiscernables au toucher.
Au moment où il s’habille, survient une panne d’électricité. Abdou, qui est pressé et n’a ni lampe de poche, ni boite d’allumettes, prend au hasard deux chaussettes dans le tiroir.
\begin{enumerate}
    \item Quelle est la probabilité pour qu’il ait tiré deux chaussettes de même couleur ?
    \item En supposant que le nombre de chaussettes vertes et le nombre de chaussettes rouges reste inchangé, quel devrait être le nombre $n$ de chaussettes noires contenues dans le tiroir pour que la probabilité d’avoir deux chaussettes noires soit égale à ?
\end{enumerate}

\section*{EXERCICE 3 : Urne avec boules}
Soit un entier $n \geq 4$. Une urne contient $n$ boules blanches et $n$ boules noires. On tire simultanément de cette urne 4 boules au hasard et on compte le nombre de boules blanches obtenues.
\begin{enumerate}
    \item Quelle est la probabilité $P_n(2)$ pour que ce nombre soit égal à 2 ?
    \item Quelle est la limite de $P_n(2)$ lorsque $n \rightarrow +\infty$ ?
\end{enumerate}

\section*{EXERCICE 4 : Pollution}
Un milieu biologique risque d’être pollué par des bactéries ou par des champignons. Ces deux sources de pollution sont indépendantes. Un milieu pollué le reste définitivement.
Au cours d’une journée d’exposition, la probabilité d’être pollué par des bactéries est 0,2 et celle d’être pollué par des champignons est 0,3.
Déterminer en fonction de $n$ la probabilité d’avoir été pollué au cours de $n$ journées d’exposition.

\section*{EXERCICE 5 : Maintenance}
Dans une entreprise, on fait appel à un technicien lors de son passage hebdomadaire, pour l’entretien des machines.
Chaque semaine, on décide donc pour chaque appareil de faire appel ou non au technicien. Pour un certain type de machines, le technicien constate :
\begin{itemize}
    \item[•] qu'il doit intervenir la première semaine ;
   \item[•] que s'il est intervenu la $n^{ième}$ semaine, la probabilité qu'il intervienne à la $(n+1)^{ième}$ semaine est égale à $p$ ;
    \item[•] que s’il n’est pas intervenu à la $n^{ième}$ semaine, la probabilité qu’il intervienne à la $(n+1)^{ième}$ semaine est égale à $q$.
\end{itemize}
On désigne par $E_n$ l’événement $\ll$ le technicien intervient la $n^{ième}$ semaine $\gg$ et par $P_n$ la probabilité de cet événement.
\begin{enumerate}
    \item Déterminer les nombres $P(E_1)$, $P(E_{n+1}/E_n)$, $P(E_{n+1}/\overline{E_n})$ puis en fonction de $P_n$ déterminer $P(E_{n+1} \cap E_n)$.
   \item En déduire que pour tout entier $n$ non nul : $P_{n+1} = P_n + p(1-P_n) + qP_n$.
   \item On pose $q_n = p_n - \frac{2}{7}$. Montrer que la suite $(q_n)$ est géométrique. En déduire l’expression de $P_n$ en fonction de $n$. Pour quelles valeurs de $n$ a-t-on $P_n \leq$ ?
\end{enumerate}
\end{document}
