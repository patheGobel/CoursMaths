\documentclass[12pt]{article}
\usepackage{stmaryrd}
\usepackage{graphicx}
\usepackage[utf8]{inputenc}

\usepackage[french]{babel}
\usepackage[T1]{fontenc}
\usepackage{hyperref}
\usepackage{verbatim}

\usepackage{color, soul}

\usepackage{pgfplots}
\pgfplotsset{compat=1.15}
\usepackage{mathrsfs}

\usepackage{amsmath}
\usepackage{amsfonts}
\usepackage{amssymb}
\usepackage{tkz-tab}
\author{\\Lycée de Dindéfelo\\Mr BA}
\title{\textbf{TD1}}
\date{\today}
\usepackage{tikz}
\usetikzlibrary{arrows, shapes.geometric, fit}

% Commande pour la couleur d'accentuation
\newcommand{\myul}[2][black]{\setulcolor{#1}\ul{#2}\setulcolor{black}}
\newcommand\tab[1][1cm]{\hspace*{#1}}

\begin{document}
\maketitle
\newpage
\section*{\underline{\textbf{\textcolor{red}{Exercice 1}}}}
Combien de menus différents peut-on composer si on a le choix entre 3 entrées, 2 plats et 4 desserts ?
\section*{\underline{\textbf{\textcolor{red}{Exercice 2}}}}
Une femme a dans sa garde-robe 4 jupes, 5 chemisiers et 3 vestes. Elle choisit au hasard une jupe, un chemisier et une veste. De combien de façons différentes peut-elle s’habiller ?
\section*{\underline{\textbf{\textcolor{red}{Exercice 3}}}}
Deux équipes de hockeys de 12 et 15 joueurs échangent une poignée de main à la fin d’un match : chaque joueur d’une équipe serre la main de chaque joueur de l’autre équipe. Combien de poignées de main ont été échangées ?
\section*{\underline{\textbf{\textcolor{red}{Exercice 4}}}}
Un questionnaire à choix multiples, autorisant une seule réponse par question, comprend 15 questions. Pour chaque question, on propose 4 réponses possibles.
De combien de façons peut-on répondre à ce questionnaire ?
\section*{\underline{\textbf{\textcolor{red}{Exercice 5}}}}
Raymond Queneau a écrit un ouvrage intitulé \textit{Cent mille milliards de poèmes}.\\
Il est composé de 10 pages contenant chacune 14 vers
Le lecteur peut composer son propre poème de 14 vers en
prenant le premier vers de l’une des 10 pages puis le deuxième
vers de l’une des 10 pages et ainsi de suite jusqu’au
quatorzième vers. Justifier le titre de l’ouvrage
\section*{\underline{\textbf{\textcolor{red}{Exercice 6}}}}
En informatique, on utilise le système binaire pour coder les caractères.
Un bit \textit{(binary digit : chiffre binaire)} est un élément qui prend la valeur 0 ou la valeur 1. Avec 8 chiffres binaires \textit{(un octet)}, combien de caractères peut-on coder ?
\section*{\underline{\textbf{\textcolor{red}{Exercice 7}}}}
Combien peut-on former de numéros de téléphone à 8 chiffres ?\\
Combien peut-on former de numéros de téléphone à 8 chiffres ne comportant pas le chiffre 0 ?
\section*{\underline{\textbf{\textcolor{red}{Exercice 8}}}}
A l’occasion d’une compétition sportive groupant 18 athlètes, on attribue une médaille d’or, une d’argent, une de bronze.\\
Combien y-a-t-il de distributions possibles (avant la compétition, bien sûr…) ?
\section*{\underline{\textbf{\textcolor{red}{Exercice 9}}}}
Un groupe d'élèves de terminale constitue le bureau de l'association\\ "L'association des Maitres de l'Absence".\\
Ce bureau est composé d'un président, d'un secrétaire et d'un trésorier.\\
Combien y a-t-il de bureaux possibles ? ( il y a 24 élèves dans la classe )
\section*{\underline{\textbf{\textcolor{red}{Exercice 10}}}}
Six personnes choisisent mentalement un nombre entier compris entre 1 et 6.\\
1) Combien de résultats peut-on obtenir ?\\
2) Combien de résultats ne comportant pas deux fois le même nombre peut-on obtenir ?
\section*{\underline{\textbf{\textcolor{red}{Exercice 11}}}}
Soit A l'ensemble des nombres de quatre chiffres, le premier étant non nul.\\
1) Calculer le nombre d'éléments de A.\\
2) Dénombrer les éléments de A :\\
a) composés de quatre chiffres distincts\\
b) composés d'au moins deux chiffres identiques\\
c) composés de quatre chiffres distincts autres que 5 et 7\\
\section*{\underline{\textbf{\textcolor{red}{Exercice 12}}}}
Un clavier de 9 touches permet de composer le code d’entrée d’un\\ immeuble, à l’aide d’une lettre suivie d’un nombre de 3 chiffres distincts ou non.\\
\begin{table}[h]
\centering
\begin{tabular}{|c|c|c|}
\hline
1 & 2 & 3 \\
\hline
4 & 5 & 6 \\
\hline
A & B & C \\
\hline
\end{tabular}
\end{table}\\
1) Combien de codes différents peut-on former ?\\
2) Combien y a-t-il de codes sans le chiffre 1 ?\\
3) Combien y a-t-il de codes comportant au moins une fois le chiffre 1 ?\\
4) Combien y a-t-il de codes comportant des chiffres distincts ?\\
5) Combien y a-t-il de codes comportant au moins deux chiffres identiques ?\\
\section*{\underline{\textbf{\textcolor{red}{Exercice 13}}}}
Le groupe des élèves de Terminale doit s'inscrire au concours par\\ EAMAC. Il faut établir une liste de passage. Combien y a-t-il de manières de constituer cette liste ? ( il y a 24 élèves dans la classe )
\section*{\underline{\textbf{\textcolor{red}{Exercice 14}}}}
Les nombres 5, -1 et 3 constituent la solution d’un système de trois\\ équations à trois inconnues. Donner tous les triplets différents qui peuvent être la solution de ce système
\section*{\underline{\textbf{\textcolor{red}{Exercice 15}}}}
Combien y-a-t-il d’anagrammes du mot MATH ?\\
\section*{\underline{\textbf{\textcolor{red}{Exercice 16}}}}
1) Dénombrer les anagrammes du mot DIEYNABA\\
2) Dans chacun des cas suivants, dénombrer les anagrammes du mot \\DIEYNABA :\\
a) commençant et finissant par une consonne ;\\
b) commençant et finissant par une voyelle ;\\
c) commençant par une consonne et finissant par une voyelle\\
d) commençant par une voyelle et finissant par une consonne\\
\section*{\underline{\textbf{\textcolor{red}{Exercice 17}}}}
Combien y-a-t-il d’anagrammes du mot TABLEAU ?
\section*{\underline{\textbf{\textcolor{red}{Exercice 18}}}}
1) Combien peut-on réaliser de mots de n lettres comportant k lettres se répétant $p_{1} ,p_{2},...,p_{k}$ fois ?\\
2) Quel est le nombre d’anagrammes du mot « MATHEMATICIENNE » ?
\section*{\underline{\textbf{\textcolor{red}{Exercice 19}}}}
Dénombrer toutes les anagrammes possibles du mot PRISÉE\\
1) En tenant compte de l’accent\\
2) En ne tenant pas compte de l’accent sur le $\ll e \gg$ 
\section*{\underline{\textbf{\textcolor{red}{Exercice 20}}}}
Un groupe de 3 élèves de Terminale doit aller chercher des livres au CDI. De combien de manières peut-on former ce groupe ?\\ (il y a 24 élèves dans la classe )
\section*{\underline{\textbf{\textcolor{red}{Exercice 21}}}}
Un tournoi sportif compte 8 équipes engagées. \\Chaque équipe doit rencontrer toutes les autres une seule fois Combien\\ doit-on organiser de matchs ?
\section*{\underline{\textbf{\textcolor{red}{Exercice 22}}}}
Au loto, il y a 49 numéros. Une grille de loto est composée de 6 de ces numéros. Quel est le nombre de grilles
différentes ?
\section*{\underline{\textbf{\textcolor{red}{Exercice 23}}}}
De combien de façons peut-on choisir 3 femmes et 2 hommes parmi 10 femmes et 5 hommes ?
\section*{\underline{\textbf{\textcolor{red}{Exercice 24}}}}
Dans une classe de 32 élèves, on compte 19 garçons et 13 filles. On doit élire deux délégués\\
1) Quel est le nombre de choix possibles ?\\
2) Quel est le nombre de choix si l’on impose un garçon et fille\\
3) Quel est le nombre de choix si l’on impose 2 garçons ?\\
\section*{\underline{\textbf{\textcolor{red}{Exercice 25}}}}
Christian et Claude font partie d’un club de 18 personnes. On doit\\ former un groupe constitué de cinq d’entre elles pour représenter le club à un spectacle.\\
1) Combien de groupes de 5 personnes peut-on constituer ?\\
2) Dans combien de ces groupes peut figurer Christian ?\\
3) Christian et Claude ne pouvant se supporter, combien de groupes de 5 personnes peut-on constituer de telle façon que Christian et Claude ne se retrouvent pas ensemble ?\\
\section*{\underline{\textbf{\textcolor{red}{Exercice 26}}}}
Au service du personnel, on compte 12 célibataires parmi les 30 employés. On désire faire un sondage : pour cela on choisit un échantillon de quatre personnes dans ce service.\\
1) Quel est le nombre d’échantillons différents possibles ?\\
2) Quel est le nombre d’échantillons ne contenant aucun célibataire ?\\
3) Quel est le nombre d’échantillons contenant au moins un célibataire ?\\
\section*{\underline{\textbf{\textcolor{red}{Exercice 27}}}}
On constitue un groupe de 6 personnes choisies parmi 25 femmes et 32 hommes\\
1) De combien de façons peut-on constituer ce groupe de 6 personnes ?\\
2) Dans chacun des cas suivants, de combien de façons peut-on constituer ce groupe avec :\\
a) uniquement des hommes ;\\
b) des personnes de même sexe ;\\
c) au moins une femme et au moins un homme\\
\section*{\underline{\textbf{\textcolor{red}{Exercice 28}}}}
On extrait simultanément 5 cartes d'un jeu de 32. Cet ensemble de 5 cartes est appelé une "main"\\
1) Combien y a-t-il de mains différentes possibles ?\\
2) Dénombrer les mains de 5 cartes contenant :\\
a) un carré\\
b) deux paires distinctes\\
c) un full (trois cartes de même valeur, et deux autres de même valeurs. Exemple : 3 rois et 2 as)\\
d) un brelan (trois cartes de même valeur, sans full ni carré\\
e) une quinte (5 cartes de même couleur, se suivant dans l'ordre croissant)\\
\section*{\underline{\textbf{\textcolor{red}{Exercice 29}}}}
Un sac contient 5 jetons verts (numérotés de 1 à 5) et 4 jetons rouges (numérotés de 1 à 4).\\
1) On tire successivement et au hasard 3 jetons du sac, sans remettre le jeton tiré. Calculer les probabilités :\\
a) De ne tirer que 3 jetons verts ;\\
b) De ne tirer aucun jeton vert\\
c) De tirer au plus 2 jetons verts ;\\
d) De tirer exactement 1 jeton vert.\\
2) On tire simultanément et au hasard 3 jetons du sac.\\
Reprendre alors les questions a), b), c) et d)
\section*{\underline{\textbf{\textcolor{red}{Exercice 30}}}}
Un portemanteau comporte 5 patères alignées.\\ 
Combien a-t-on de dispositions distinctes (sans mettre deux manteaux l’un sur l’autre) :\\
a) pour 3 manteaux sur ces 5 patères ?\\
b) pour 5 manteaux ?\\
c) pour 6 manteaux ?\\
\section*{\underline{\textbf{\textcolor{red}{Exercice 31}}}}
Quatre garçons et deux filles s’assoient sur un banc.\\
1) Quel est le nombre de dispositions possibles ?\\
2) Même question si les garçons sont d’un côté et les filles de l’autre.\\
3) Même question si chaque fille est intercalée entre deux garçons.\\
4) Même question si les filles veulent rester l’une à côté de l’autre.\\
\newpage
\begin{center}
\section*{\underline{\textbf{\textcolor{red}{Correction}}}}
Think first!!
\end{center}

\end{document}