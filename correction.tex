\documentclass[12pt]{article}
\usepackage{stmaryrd}
\usepackage{graphicx}
\usepackage[utf8]{inputenc}

\usepackage[french]{babel}
\usepackage[T1]{fontenc}
\usepackage{hyperref}
\usepackage{verbatim}

\usepackage{color, soul}

\usepackage{pgfplots}
\pgfplotsset{compat=1.15}
\usepackage{mathrsfs}

\usepackage{amsmath}
\usepackage{amsfonts}
\usepackage{amssymb}
\usepackage{tkz-tab}
\author{\\Lycée de Dindéfelo\\Mr BA}
\title{\textbf{Correction}}
\date{\today}
\usepackage{tikz}
\usetikzlibrary{arrows, shapes.geometric, fit}

% Commande pour la couleur d'accentuation
\newcommand{\myul}[2][black]{\setulcolor{#1}\ul{#2}\setulcolor{black}}
\newcommand\tab[1][1cm]{\hspace*{#1}}

\begin{document}
\maketitle
\newpage
\section*{\underline{\textbf{\textcolor{red}{Exercice 1}}}}
Combien de menus différents peut-on composer si on a le choix entre 3 entrées, 2 plats et 4 desserts ?
\section*{\underline{\textbf{\textcolor{red}{Correction}}}}
Pour calculer le nombre de menus différents, nous utilisons le principe de multiplication.

Si nous avons 3 choix d'entrées, 2 choix de plats et 4 choix de desserts, le nombre total de menus différents est le produit de ces choix.

Nombre de menus différents = (nombre d'entrées) × (nombre de plats) ×
	 (nombre de desserts)

En substituant les valeurs, nous avons :

Nombre de menus différents = 3 × 2 × 4

Calculons le produit :

Nombre de menus différents = 24

Donc, il est possible de composer 24 menus différents en choisissant parmi 3 entrées, 2 plats et 4 desserts.
\section*{\underline{\textbf{\textcolor{red}{Exercice 2}}}}
Une femme a dans sa garde-robe 4 jupes, 5 chemisiers et 3 vestes. Elle choisit au hasard une jupe, un chemisier et une veste. De combien de façons différentes peut-elle s’habiller ?
\section*{\underline{\textbf{\textcolor{red}{Correction}}}}
Pour calculer le nombre de façons différentes qu'elle peut s'habiller, nous utilisons également le principe de multiplication.

Elle a le choix entre 4 jupes, 5 chemisiers et 3 vestes. Le nombre total de façons différentes qu'elle peut s'habiller est le produit de ces choix.

Nombre de façons différentes = (nombre de jupes) × (nombre de chemisiers) × (nombre de vestes)

En substituant les valeurs, nous avons :

Nombre de façons différentes = 4 × 5 × 3

Calculons le produit :

Nombre de façons différentes = 60

Donc, elle peut s'habiller de 60 façons différentes en choisissant une jupe, un chemisier et une veste au hasard dans sa garde-robe.
\section*{\underline{\textbf{\textcolor{red}{Exercice 3}}}}
Deux équipes de hockeys de 12 et 15 joueurs échangent une poignée de main à la fin d’un match : chaque joueur d’une équipe serre la main de chaque joueur de l’autre équipe. Combien de poignées de main ont été échangées ?
\section*{\underline{\textbf{\textcolor{red}{Correction}}}}
Pour résoudre ce problème, je t'invite à utiliser le principe multiplicatif. Celui-ci te permettra de déterminer le nombre total de façons différentes en multipliant le nombre d'options disponibles pour chaque élément que tu dois choisir.
\end{document}