\documentclass[12pt]{article}
\usepackage{stmaryrd}
\usepackage{graphicx}
\usepackage[utf8]{inputenc}
\usepackage[french]{babel}
\usepackage[T1]{fontenc}
\usepackage{hyperref}
\usepackage{verbatim}
\usepackage{color,soul}
\usepackage{amsmath}
\usepackage{amsfonts}
\usepackage{amssymb}
\usepackage{systeme}
\usepackage{tkz-tab}
\author{Destiné à la TerminaleS2\\Au Lycée de Dindéferlo}
\title{\textbf{Fonction Exponentielle}}
\date{\today}
\usepackage{tikz}
\usetikzlibrary{arrows}
\usepackage[a4paper,left=20mm,right=20mm,top=15mm,bottom=15mm]{geometry}
\usepackage{mathtools}
\usepackage{systeme}

\usepackage{pgfplots}
\pgfplotsset{compat=1.15}
\usepackage{mathrsfs}
\usetikzlibrary{arrows}
\pagestyle{empty}

\DecimalMathComma

\begin{document}

\maketitle
\newpage
\subsection*{\underline{\textbf{\textcolor{red}{Exercice 1:}}}}
1.Résoudre dans $\mathbb{R}$ les équations et inéquations suivantes :\\
1. $(e^{x}-7)(e^{x}+4)=0$\quad\quad 2. $e^{\ln(1-x^{2}}=-2x+1$\quad\quad 3. $(e^{x}-7)(e^{x}+4)\geq 0$\\
1. $2e^{2x}-11e^{x}+15=0$\quad\quad 2. $(x^{2}-1)e^{\ln(x-2)}={\ln}e^{x+1}$\quad\quad 
3. $5e^{2x}+7e^{x}-6<0$\\
\section*{\underline{\textbf{\textcolor{red}{Exercice 2:}}}}
1. Déterminer la fonction dérivée de la fonction $f$. On précisera l’ensemble de définition de $f$ et celui de sa fonction dérivée $f'$\\
1.$f(x)=e^{x^{2}}$\quad\quad\quad  4.$f(x)=e^{3x}-6e^{2x}+9e^{x}<0$\quad\quad\quad  
7.$\frac{e^{x}-1}{e^{x}+2}$\\
2.$f(x)=e^{2x}-2e^{x}$\quad\quad\quad  5.$f(x)=(x+3)e^{-x}$\quad\quad\quad  
8.$f(x)=\frac{2e^{x}-1}{e^{x}-1}$\\
3.$f(x)=(e^{x}-1)^{2}$\quad\quad\quad  6.$f(x)=x+4e^{-x}$\quad\quad\quad  
9.$f(x)=xe^{\frac{1}{x+3}}$\\
Déterminer les primitives sur $\mathbb{R}$ de la fonction $f$\\
1.$f(x)=e^{3x}$\quad\quad 2.$f(x)=xe^{x^{2}}$\quad\quad 3.\(f(x)=\cos(x)e^{{\sin(x)}}\)\quad
4.$f(x)=\frac{e^{x}}{e^{x}+1}$
\subsection*{\underline{\textbf{\textcolor{red}{Exercice 3:}}}}
Calculer les limites suivantes :\\
\[1.\lim_{x \to +\infty}(e^{x}-1)\quad\quad 2.\lim_{x \to -\infty}(e^{x}-1)\quad\quad  3.\lim_{x \to +\infty}\frac{e^{x}-1}{e^{2x}+2}\quad\quad 4.\lim_{x \to -\infty}(e^{2x}-e^{x}) \quad\quad 5. \lim_{x \to 0}\frac{e^{x^{2}}-1}{1-\cos(x)}\]
\[6.\lim_{x \to 0} e^{x}-x\ln|x|\quad\quad 7.\lim_{x \to -\infty} e^{x}-x\ln|x| 
\quad\quad 8.\lim_{x \to +\infty}\frac{e^{2x}+1}{e^{x}-3}\quad\quad
9.\lim_{x \to +\infty}\ln\vert\frac{e^{2x}+5}{e^{x}-2}\vert\]
\[10.\lim_{x \to -\infty}{xe^{-\frac{1}{x^{2}}}}\quad\quad
11.\lim_{x \to 0}\frac{{e^{3x}-1}}{2x}\quad\quad 12.\lim_{x \to +\infty} x-\ln(1+e^{x})\quad\quad 13.\lim_{x \to +\infty} \ln(e^{x}-1)-x\]
\[14.\lim_{x \to 1^{+}}(x-1)e^{\frac{1}{x-1}}\quad\quad 
15.\lim_{x \to 0}\frac{e^{3x}-e^{2x}}{x}\]
\section*{\underline{\textbf{\textcolor{red}{Exercice 4: }}}}
I.Soit la fonction $f$ définie par $f(x)=e^{x}+x+1$\\
1. Étudier les variations de $f$.\\
2. Montrer que l’équation $f (x) = 0$ admet une solution unique $\alpha$ dans $\mathbb{R}$.\\
Montrer que $-1,28 \leq \alpha \leq -1,27.$\\
3. En déduire le signe de $f (x)$ sur $\mathbb{R}$.\\
II.Soit $g(x)=\frac{xe^{x}}{e^{x}+1}$ On note $C_{g}$ la courbe représentative de $g$ dans un repère orthonormé.\\
1. Montrer que $g'(x)=\frac{e^{x}f(x)}{(e^{x}+1)^{2}}$\\
En déduire le sens de variation de $g$ sur $\mathbb{R}$.\\
2. Montrer que $g(\alpha) = \alpha + 1$.\\
En déduire un encadrement de $g(\alpha)$.\\
3. Soit $(T)$la tangente à $C_{g}$ en un point d’abscisse $0$.\\
Étudier la position de $C_{g}$ par rapport à $(T)$.\\
4. Démontrer que la droite $(D) : y = x$ est asymptote à $C_{g}$\\
Étudier la position de $C_{g}$ par rapport à $(D)$.\\
5. Dresser le tableau de variation de g.\\
Tracer $C_{g}$, $(T)$ et $(D)$ dans le repère $(O;\vec{i};\vec{j})$.
\subsection*{\underline{\textbf{\textcolor{red}{Exercice 5:}}}}
On considère la fonction $f$ définie sur $\mathbb{R}$ par :\\
\[ f(x) = \begin{cases} 
  2x+(1-x)e^{x}, & \text{si } x < 0 \\
  xe^{-2x}, & \text{si } \geq 0
\end{cases} \]
et $C_{f}$ sa courbe dans un repère orthonormé $(O,\vec{i},\vec{j})$ d’unité 2 cm.\\
1. Déterminer l’ensemble de définition de f.\\
2. (a) Etudier les limites de $f$ en $-\infty$ et en $+\infty$.\\
(b) Interpréter graphiquement, si possible, les résultats obtenus.\\
(c) Montrer que la droite $(\delta)$ d’équation $y = 2x$ est\\
une asymptote oblique à $C_{f}$ en $-\infty$.\\
3. Etudier la continuité de $f$ en 0.\\
4. Soit $h$ la fonction définie sur $\left] -\infty;0  \right[$ par : $h(x) = 2 - xe^{x}$\\
(a) Dresser le tableau de variations de $h$ sur $\left]-\infty; 0\right[ $.\\
(b) En déduire son signe sur $\left] -\infty; 0\right[$ .\\
5. (a) Etudier la dérivabilité de $f$ en $0$.\\
(b) Interpréter graphiquement, si possible, les résultats obtenus.\\
(c) Calculer $f'(x)$ dans chaque intervalle où $f$ est dérivable\\
(d) Dresser le tableau de variations de $f$ sur $\mathbb{R}$.\\
6. tracer la courbe $C_{f}$ dans le repère $(O,\vec{i},\vec{j})$.\\
\section*{\underline{\textbf{\textcolor{red}{Exercice 6:}}}}
Soit f la fonction définie par
\[ f(x) = \begin{cases} 
  1-\ln(x^{2}+1), & \text{si } x \leq 0 \\
  -x^{2}+e^{-x}, & \text{si } > 0
\end{cases} \]
On désigne par $C_{f}$ sa courbe représentative.\\
1. Etudier la dérivabilité de $f$ en 0.\\
2. Etudier les branches infinies de $C_{f}$; démontrer que la parabole $(\Gamma)$ d’équation $y=-x^{2}$ est asymptote à $C_{f}$
en $+\infty$.\\
3. Compléter l’étude de f et construire $(\Gamma)$ et $C_{f}$.\\
4. (a) Déduire de cette étude que $C_{f}$ coupe l’axe des abscisses en deux points dont l’un a une abscisse négative que l’on calculera.\\
(b) Déterminer une valeur approchée à $10^{-1}$ près de\\ l’abscisse du deuxième point d’intersection.\\
\subsection*{\underline{\textbf{\textcolor{red}{Exercice 7:}}}}
I. On considère la fonction $g$ définie sur $\mathbb{R}^{*}$ par :\\
$g(x)=(1+\frac{1}{x})e^{\frac{1}{x}}+1$\\
1. Déterminer les limites aux bornes de l’ensemble de définition de g.\\
2. Dresser le tableau de variations de g.\\
3. En déduire que $\forall x \in \mathbb{R}^{*}, g(x)>0.$
Soit f la fonction définie par :\\
\[ \begin{cases} 
  f(x)=\frac{x}{1+e^{\frac{1}{x}}}, & \text{si } x < 0 \\
  fx)=x[1-\ln(x)]^{2}, & \text{si } > 0 \\
  f(0)=0
\end{cases} \]
On désigne par (Cf ) sa courbe représentative dans un repère orthonormé 
$(O,\vec{i},\vec{j})$. (unité 2 cm).\\
1. Justifier que $f$ est définie sur $\mathbb{R}$.\\
2. Calculer les limites aux bornes de $D_{f}$ .\\
3. Etudier la continuité et la dérivabilité de f en 0. Interpréter les résultats.\\
4. Montrer que la droite (D) d’équation $y=\frac{1}{2}x-\frac{1}{4}$ est asymptote à $C_{f}$ en $-\infty$. (On pourra poser $X=\frac{1}{x}$\\
5. Etudier la nature de la branche infinies en $+\infty$.\\
6. Montrer que $\forall x \in \left] -\infty; 0\right[$ , $f'(x)=\frac{g(x)}{(1+e^{\frac{1}{x}})^{2}}$\\
7. Dresser le tableau de variation de f.\\
8. Construire (Cf ) dans le repère $(O,\vec{i},\vec{j})$.\\
III.Soit $h$ la restriction de $f$ à l’intervalle $I=\left[ e;+\infty\right[ $\\
1. Montrer que h réalise une bijection de $I$ vers un intervalle $J$ à préciser. Dresser le tableau de variation de $h^{-1}$, la bijection réciproque de $h$.\\
2. Etudier la dérivabilité de $h^{-1}$ sur $J$.\\
3. Calculer $h(e^{2})$ puis $(h^{-1})'(e^{2})$\\
4. Construire $(C_{h^{-1}})$, la courbe de $h^{-1}$ dans le repère $(O,\vec{i},\vec{j})$\\
\section*{\underline{\textbf{\textcolor{red}{Exercice 8: }}}}
Soit f la fonction de la variable réelle x définie par :\\
$f(x)=\frac{e^{x}}{e^{x}+1}-\ln(1+e^{x})$\\
1. (a) Étudier les variations de f.\\
(b) Montrer que $\lim_{x \to +\infty} \left[f(x)-1+x \right]=0$. Que peut-on en déduire pour la courbe représentative de $f$?\\
Tracer cette courbe (unité : 2cm).\\
(c) Montrer que f réalise une bijection de $ \left] -\infty ; +\infty \right[ $
vers $\left] -\infty ; 0\right[ $.\\
2. Soit g la fonction de la variable réelle x définie par :\\
$g(x)=e^{-x}\ln(1+e^{x})$.\\
(a) Démontrer que g est dérivable sur $\mathbb{R}$.\\
(b) Montrer que quel que soit le réel x, $g'(x)=e^{-x}f(x)$.\\
(c) Montrer que :\[\lim_{x \to +\infty}g(x)=0\quad et \lim_{x \to -\infty}g(x)=1\]
(d) Étudier les variations de g et tracer sa courbe représentative dans le repère précédent.\\
3. Montrer que $g$ est une bijection de $\mathbb{R}$ sur un intervalle J à préciser.\\
4. (a) Calculer $g(0)$.\\
(b) Montrer que $g^{-1}$ est dérivable en ln(2).\\
(c) Déterminer l’équation de la tangente à $C^{-1}_{g}$ au point d’abscisse ln(2).
\subsection*{\underline{\textbf{\textcolor{red}{Exercice 9:}}}}
\[ \begin{cases} 
  f(x)=\frac{1}{x^{2}}e^{\frac{1}{x}}, & \text{si } x < 0 \\
  fx)=x^{2}\ln(x), & \text{si } > 0 \\
  f(0)=0
\end{cases} \]
1. (a) Étudier la continuité de $f$ en 0.\\
(b) Étudier la dérivabilité de $f$ en 0. Interpréter les
résultats.\\
2. (a) Calculer les limites de $f$ en $-\infty$ et en $+\infty$.\\
(b) Calculer $f'(x$), on précisera le domaine de dérivabilité de $f$.\\
(c) Étudier le signe de $f'(x)$ et établir le tableau de variation de $f$.\\
3. Étudier les branches infinies de la courbe de $f$.\\
4. Tracer la courbe représentative de $f$ dans le plan muni du repère orthonormé 
$(O,\vec{u},\vec{v})$\\ (unité : 4 cm).\\
\section*{\underline{\textbf{\textcolor{red}{Exercice 10: }}}}
\textbf{Partie A}\\ 
1. Étudier sur $\mathbb{R}$ le signe de $4e^{2x}-5e^{x}+1.$\\
2. Soit $\varphi$ la fonction définie par : $\varphi(x)=\ln(x)-2\sqrt{x}+2.$\\
(a) Déterminer son domaine de définition $D_{\varphi}$ et calculer ses limites aux bornes de $D_{\varphi}$\\
(b) Étudier ses variations et dresser son tableau de variations.\\
(c) En déduire son signe.\\
\textbf{Partie B}\\
Soit f la fonction définie par :
\[f(x)=\begin{cases} 
  x+\frac{e^{x}}{2e^{x}-1}, & \text{si } x \leq 0 \\
  1-x+\sqrt{x}\ln(x), & \text{si } > 0 
\end{cases} \]
On désigne par $C_{f}$ la courbe représentative de f dans un repère orthonormé d’unité 2 cm\\
1. (a) Déterminer $D_{f}$ le domaine de définition de $f$.\\
(b) Déterminer les limites de $f$ aux bornes de $D_{f}$ et étudier les branches infinies de $C_{f}$ .\\
(c) Étudier la position de $C_{f}$ par rapport à l’asymptote non parallèle aux axes dans 
$\left] -\infty, 0\right] $\\
2. (a) Étudier la continuité de $f$ en 0.\\
(b) Étudier la dérivabilité de f en 0 et interpréter graphiquement les résultats.\\
3. Déterminer la dérivée de f et dresser le tableau de variations de f.\\
4. Construire dans le repère les asymptotes, la courbe $C_{f}$ et les demi-tangentes. On remarquera que $f(1) = 0$ et $f'(1) = 0$\\
\subsection*{\underline{\textbf{\textcolor{red}{Exercice 11:}}}}
\section*{\underline{\textbf{\textcolor{red}{Exercice 12: }}}}
\subsection*{\underline{\textbf{\textcolor{red}{Exercice 13:}}}}
\section*{\underline{\textbf{\textcolor{red}{Exercice 14: }}}}
\subsection*{\underline{\textbf{\textcolor{red}{Exercice 15:}}}}
\section*{\underline{\textbf{\textcolor{red}{Exercice 16: }}}}
\end{document}
